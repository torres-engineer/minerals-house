\chapter{Conclusão}
\label{ch:conclusion}

O projeto \textbf{Casa dos Minerais} ainda se encontra em fase de
planeamento e desenvolvimento, mas já tem uma base sólida e bem
definida. A ideia principal é criar um jogo educativo acessível e
divertido, que ajude as pessoas a aprender mais sobre os minerais e a
sua importância no nosso dia a dia.

Durante esta fase, foi possível definir os principais objetivos, as
tecnologias que vamos usar e a forma como o jogo poderá ser
desenvolvido. Também pensamos em aspetos importantes como a
acessibilidade, a utilização de software livre e a importância de
tornar o jogo disponível para diferentes públicos.

De forma geral, o projeto tem potencial para crescer e evoluir,
servindo como ferramenta educativa e de sensibilização. O próximo
passo será avançar com o desenvolvimento prático e começar a
transformar as ideias em realidade.

\section{Futuro}
\label{sec:future}

O ``Casa dos Minerais'' está a ser planeado como um projeto escalável.
Embora a versão inicial se foque na experiência single-player e na
exploo como fea casa, a equipa já identificou várias expansões futuras
que podem enriquecer significativamente a vertente pedagógica, a
acessibilidade e a rejogabilidade.

Para além da potencial exploração de um modo \textbf{multijogador}
(que poderia assumir a forma de questionários competitivos ou
exploração cooperativa de um mapa), destacam-se as seguintes funcionalidades:

\begin{enumerate}
  \item {
      Enciclopédia de Minerais (O ``Mineral-dex'')
      \begin{longtable}{p{.3\textwidth}p{.7\textwidth}}
        Conceito & Implementar um ``caderno de campo'' ou
        enciclopédia digital.\\
        Funcionalidade & Cada vez que o jogador descobre um novo
        mineral num objeto, este seria automaticamente adicionado à
        enciclopédia. O jogador poderia consultá-la a qualquer
        momento para rever todos os minerais que já encontrou, as
        suas propriedades, proveniência, e uma lista dos objetos onde
        estes estão presentes (ex: ``Silício: encontrado nos
        microchips da TV e do telemóvel'').
      \end{longtable}
    }
  \item {
      Níveis de Dificuldade Selecionáveis
      \begin{longtable}{p{.3\textwidth}p{.7\textwidth}}
        Conceito & Alinhar o jogo ainda mais com os objetivos de
        acessibilidade, permitindo ao jogador adaptar a experiência
        às suas necessidades ou objetivos.\\
        Modos Proposto & {
          \begin{itemize}
            \item Modo Exploração (Fácil): Focado puramente na
              descoberta. Os questionários seriam opcionais ou
              inexistentes, e não haveria temporizadores. Seria ideal
              para públicos mais jovens ou para uma utilização livre
              em sala de aula.
            \item Modo Desafio (Normal): A experiência principal como
              definida, onde a conclusão do questionário de nível é
              obrigatória para progredir.
            \item Modo Contra-relógio (Difícil): Ativaria o sistema
              de pontuação baseado no tempo, oferecendo um desafio
              adicional para quem procura rejogar os níveis.
          \end{itemize}
        }
      \end{longtable}
    }
  \item {
      Sistema de Pistas (Hint System) nos Questionários
      \begin{longtable}{p{.3\textwidth}p{.7\textwidth}}
        Conceito &  Integrar um sistema de ajuda nos minijogos de
        questionário para evitar a frustração e manter o foco na aprendizagem.\\
        Funcionalidade & O jogador poderia usar ``pontos'' (ganhos
        durante a exploração ou ao descobrir novos minerais) para
        pedir uma pista durante o questionário. As pistas poderiam
        incluir ``eliminar uma resposta errada''.
      \end{longtable}
    }
\end{enumerate}
