
\chapter{Introdução}
\label{ch:intro}

«Os minerais são a matéria"-prima essencial para a esmagadora
maioria dos objetos que utilizamos no nosso quotidiano, desde os
dispositivos eletrónicos mais complexos aos materiais de construção
mais básicos. No entanto, a origem e a importância destes recursos
são frequentemente desconhecidas do público em geral.» Não
interresa se estás a ler este documento num dispositivo eletrónico,
ou em impresso, os minerais, eles estão presentes tanto nos
\textit{microchips} (e não só) assim como no papel também.

Faz sentido que nunca pensemos na origem dos bens que usamos no
nosso dia"-a"-dia. Eles aparecem \textit{por magia} nas parteleiras
das lojas, tu os compras, e os usas para aquilo que tu precisas sem
pensar muito em todo o processo que foi necessário para esse
produto existir. Se tu já cozinhaste, mais depressa olhas para uma
refeição e consegues ter uma ideia de quais foram os ingredientes
necessários para confécioná"-lo. Bem, imagino que o leitor
provavelmente nunca fez parte do processo da criação de uma resma
de papel\footnote{Que também não é com certeza um processo simples e
  linear. É necessário ferramentas e máquinas para cortar as àrvores
  e processar a madeira, e essas ferramentas e máquinas têm que ser
  feitas anteriormente, talvez por outras ferramentas e máquinas, e
  por aí adiante, e essas ferramentas, máquinas e tudo o que é
necessário durante o processo, usa minerais.}, nem da construção
das peças de um computador, nem dos processos anteriores a esse
para as peças das peças existirem, \textbf{incluindo a extração de
  minérios do solo. Não sabe de onde os minérios provêm, e por que
razão eles são um ingrediente necessário} para confecionar o produto final.

Um painel solar, uma televisão, uma telha ou uma torneira, todos
estes e muitos mais, normalmente presentes nas próprias casas e
sendo objetos que usamos no dia"-a"-dia, para o seu fabrico, são
necessários vários minerais essenciais. Às vezes são coisas que nem
chegamos a dar o valor que elas realmente possuem, exatamente
porque o processo de produção está parcial ou completamente
escondido do consumidor. Não pensamos, por exemplo, na quantidade
de trabalhadores de várias zonas do globo que fazem o seu trabalho
às vezes nas condições mais precárias, para extrair o minério da
natureza usando a sua força humana. Outra vez, o que quer que seja
que está a tocar o teu pé agora (meia, calçado, o chão da tua casa,
etc., etc.), por exemplo, não apareceu magicamente na tua vida
porque alguém teve dinheiro para o comprar, existe toda uma cadeia de
trabalho humano envolvida.

\section{Objetivos}
\label{sec:objectives}

\begin{itemize}
  \item Promover a literacia científica sobre os minerais e a sua
    importância no quotidiano;
  \item Usar a \textbf{gamificação} como uma ferramenta pedagógica
    para tornar «a aprendizagem em experiências interativas e
    motivadoras», mais envolventes e visualmente apelativas a
    criação de um jogo educativo e recreativo.
\end{itemize}

No jogo, através da exploração de uma casa virtual, o jogador
poderá clicar em diferentes objetos do dia"-a"-dia como uma
televisão, um frigorifico ou uma torneira, e descobrir os minerais
essenciais para o seu fabrico, a sua proveniência e relevância.

% deve ficar claro porque é que é importante/útil resolver esse problema

\section{Estrutura do documento}

O relatório está organizado da seguinte forma. No
Capítulo~\ref{ch:intro} é apresentado o contexto do projeto e
definidos os objetivos, bem como a forma como o documento se encontra
estruturado.

O Capítulo~\ref{ch:develop} reúne o enquadramento teórico e
pedagógico que suporta o jogo, descreve a metodologia seguida durante
o trabalho e explica a pilha tecnológica escolhida.

No Capítulo~\ref{ch:implementacao} é descrita a concretização técnica
do projeto, incluindo a arquitetura geral, o modelo de dados, a
construção do mapa e do motor gráfico, o sistema de navegação e a
interface com o utilizador.

Por fim, o Capítulo~\ref{ch:conclusao} sintetiza os principais
resultados alcançados, discute as limitações atuais do protótipo e
identifica linhas de investigação e desenvolvimento futuro.

\section{Contexto e motivação educativa}

Vidro, cerâmica, cimento, metais, componentes eletrónicos, todos
dependem de minerais específicos, extraídos e processados em cadeias
de produção espalhadas pelo mundo. Apesar disso, a maioria das
pessoas nunca pensa na origem geológica dos objetos que usa. Esta
distância entre o quotidiano e a geologia dificulta a compreensão de
temas como sustentabilidade, reciclagem ou o impacto ambiental da
extração mineira.

O jogo parte de uma ideia simples: em vez de apresentar a informação
de forma expositiva, o utilizador explora cenários familiares —
começa numa casa, depois vamos para uma escola, etc., etc. — e
descobre por si próprio quais os minerais presentes nos objetos à sua
volta. O facto de serem ambientes conhecidos ajuda a tornar conceitos
como ``recurso mineral'' ou ``processamento'' em algo concreto.

Depois da exploração, o utilizador responde a um questionário curto
de seis perguntas, que serve para consolidar o que observou. Não há
uma ordem fixa de interação, cada pessoa explora ao seu ritmo e
escolhe os objetos com que quer interagir. A exploração introduz, o
questionário reforça, e o ciclo pode repetir-se.

\section{Cenário de uso}

A Casa dos Minerais pode ser usado como recurso complementar em
disciplinas de geologia, ciências naturais, materiais ou tecnologia.
Não exige conhecimentos prévios, o que o torna acessível também a público geral.

O cenário mais direto é a sala de aula: o professor sugere que os
alunos explorem o cenário, identifiquem objetos e consultem a
informação associada, e depois respondam ao questionário de seis
perguntas. Mas o jogo também funciona fora da escola, em sessões
curtas e autónomas — o fluxo é simples o suficiente para isso.

Como corre no navegador, não precisa de instalação de um software em
específico e funciona em dispositivos diferentes, o que é útil em
escolas com recursos limitados. Pode ser usado individualmente ou em
pequenos grupos.
