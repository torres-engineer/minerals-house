\chapter{Introdução}
\label{ch:intro}


«Os minerais são a matéria"-prima essencial para a esmagadora
maioria dos objetos que utilizamos no nosso quotidiano, desde os
dispositivos eletrónicos mais complexos aos materiais de construção
mais básicos. No entanto, a origem e a importância destes recursos
são frequentemente desconhecidas do público em geral.» Não
interresa se estás a ler este documento num dispositivo eletrónico,
ou em impresso, os minerais, eles estão presentes tanto nos
\textit{microchips} (e não só) assim como no papel também.

Faz sentido que nunca pensemos na origem dos bens que usamos no
nosso dia"-a"-dia. Eles aparecem \textit{por magia} nas parteleiras
das lojas, tu os compras, e os usas para aquilo que tu precisas sem
pensar muito em todo o processo que foi necessário para esse
produto existir. Se tu já cozinhaste, mais depressa olhas para uma
refeição e consegues ter uma ideia de quais foram os ingredientes
necessários para confécioná"-lo. Bem, imagino que o leitor
provavelmente nunca fez parte do processo da criação de uma resma
de papel\footnote{Que também não é com certeza um processo simples e
  linear. É necessário ferramentas e máquinas para cortar as àrvores
  e processar a madeira, e essas ferramentas e máquinas têm que ser
  feitas anteriormente, talvez por outras ferramentas e máquinas, e
  por aí adiante, e essas ferramentas, máquinas e tudo o que é
necessário durante o processo, usa minerais.}, nem da construção
das peças de um computador, nem dos processos anteriores a esse
para as peças das peças existirem, \textbf{incluindo a extração de
  minérios do solo, de onde eles provêm, e por que razão eles são um
ingrediente necessário} para confecionar o produto final.

Um painel solar, uma televisão, uma telha ou uma torneira, todos
estes e muitos mais, normalmente presentes nas próprias casas e
sendo objetos que usamos no dia"-a"-dia, para o seu fabrico, são
necessários vários minerais essenciais. Às vezes são coisas que nem
chegamos a dar o valor que elas realmente possuem, exatamente
porque o processo de produção está parcial ou completamente
escondido do consumidor. Não pensamos, por exemplo, na quantidade
de trabalhadores de várias zonas do globo que fazem o seu trabalho
às vezes nas condições mais precárias, para extrair o minério da
natureza usando a sua força humana. Outra vez, o que quer que seja
que está a tocar o teu pé agora (meia, calçado, o chão da tua casa,
etc., etc.), por exemplo, não apareceu magicamente na tua vida
porque alguém teve dinheiro para o comprar, existe toda uma cadeia de
trabalho humano envolvida.

\section{Objetivos}
\label{sec:objectives}

\begin{itemize}
  \item Promover a literacia científica sobre os minerais e a sua
    importância no quotidiano;
  \item Usar a \textbf{gamificação} como uma ferramenta pedagógica
    para tornar «a aprendizagem em experiências interativas e
    motivadoras», mais envolventes e visualmente apelativas a
    criação de um jogo educativo e recreativo.
\end{itemize}

% No jogo, através da exploração de uma casa virtual, o jogador
% poderá clicar em diferentes objetos do dia"-a"-dia como uma
% televisão, um frigorifico ou uma torneira, e descobrir os minerais
% essenciais para o seu fabrico, a sua proveniência e relevância.

% deve ficar claro porque é que é importante/útil resolver esse problema

\section{Estrutura do documento}
O relatório está organizado da seguinte forma. No Capítulo 1 (Introdução) 
é apresentado o contexto do projeto e definidos os objetivos, 
bem como a forma como o documento se encontra estruturado. 
O Capítulo 2 (Desenvolvimento) reúne o enquadramento teórico e pedagógico que suporta o jogo, 
descreve a metodologia seguida durante o trabalho e explica a pilha tecnológica escolhida.

No Capítulo 3 (Implementação) é descrita a concretização técnica do projeto, 
incluindo a arquitetura geral, o modelo de dados, a construção do mapa e do motor gráfico, 
o sistema de navegação e a interface com o utilizador.

Por fim, o Capítulo 4 (Conclusão e Trabalho Futuro) sintetiza os principais resultados alcançados, discute as limitações atuais do protótipo e identifica linhas de investigação e desenvolvimento futuro. 

\section{Contexto e motivação educativa}
Os minerais estão presentes em praticamente todos os aspetos do quotidiano, 
ainda que a sua origem e função passem muitas vezes despercebidas.
Materiais comuns como vidro, cerâmica, cimento, metais e componentes eletrónicos dependem de recursos minerais específicos, 
extraídos, processados e integrados em cadeias de produção globais.
Esta “invisibilidade” do contributo geológico no dia a dia contribui para uma literacia limitada sobre matérias-primas, 
dificultando a compreensão de temas atuais como sustentabilidade, 
reciclagem, dependência tecnológica e impacto ambiental associado à extração.
Neste contexto, o projeto surge com a resposta pedagógica baseada em aprendizagem ativa: em vez de apresentar a informação de forma expositiva, 
o jogo promove a exploração de cenários familiares (inicialmente uma casa e, numa fase seguinte, uma escola), 
onde o utilizador identifica objetos e descobre os minerais associados ao seu fabrico. 
A escolha de ambientes reconhecíveis reduz a barreira de entrada e facilita a ligação entre conceitos abstratos (mineral, recurso, processamento) e exemplos concretos do quotidiano.
A motivação educativa do jogo assenta em três objetivos principais. Primeiro, contextualizar a geologia no quotidiano, 
associando objetos comuns a minerais específicos para tornar a aprendizagem mais concreta e memorável. 
Segundo, reforçar a retenção por avaliação formativa: 
após a exploração, o utilizador realiza um questionário curto de seis perguntas, 
consolidando o conhecimento através de recordação ativa. Terceiro, 
promover curiosidade e autonomia, permitindo que o utilizador explore ao seu ritmo e escolha os objetos com que interage, 
sem impor uma sequência rígida.

Desta forma, o jogo procura equilibrar a componente informativa e a componente recreativa: 
a exploração funciona como introdução e descoberta, 
enquanto o questionário atua como mecanismo de verificação e reforço, 
permitindo repetir o ciclo e consolidar progressivamente os conteúdos.

\section{Cenário de uso}

O Casa dos Minerais foi concebido para adequação a contextos educativos. 
Podendo ser utilizado como recurso complementar em disciplinas onde existam conteúdos relacionados com geologia, 
recursos naturais, materiais e tecnologia. 
A abordagem visual e baseada em interação permite também que seja acessível a utilizadores sem conhecimentos prévios aprofundados, 
incluindo público geral interessado em compreender a origem dos materiais presentes no quotidiano.

O caso de uso mais natural é o ambiente de sala de aula, 
como atividade curta de introdução ou consolidação de conteúdos. 
O professor pode sugerir que os alunos explorem o cenário (casa e a escola), 
identifiquem objetos e consultem a informação associada, 
seguindo-se um questionário de seis perguntas para reforço imediato. 
Em alternativa, o jogo pode ser usado de forma autónoma fora do contexto escolar, 
em sessões curtas, devido ao seu fluxo simples e repetível que permite aprender por descoberta e reforço progressivo.

Como o jogo foi pensado para execução no navegador, 
o acesso é simples e compatível com diferentes dispositivos, 
o que facilita o seu uso em ambientes com recursos limitados. 
Esta característica é relevante tanto para trabalho individual 
como para atividades em pequenos grupos, onde o objetivo 
é promover curiosidade, discussão e ligação entre objetos comuns e os minerais necessários à sua produção.


