\chapter{Introdução}
\label{intro}

% deve substituir as duas linhas seguintes pelo texto da introdução

Sugestão de capítulos e/ou secções para relatórios de projetos ou
dissertações de mestrado:
\begin{enumerate}
  \item Estado da arte/trabalho relacionado (o que existe / o que se
    sabe sobre o problema a resolver); é um capítulo teórico que pode
    apresentar o que importa saber sobre o problema;
  \item  Descrição dos métodos para resolver o problema; pode incluir
    uma descrição das tecnologias, ferramentas e linguagens
    utilizadas, se foi feito um inquérito, etc.;
  \item Um ou mais capítulos/secções sobre o que foi feito; por
    exemplo, análise, projecto, implementações e teste/validação;
  \item Discussão / Conclusão (podem ser capítulos separados)
\end{enumerate}

Importa ter presente que o principal objetivo do documento é o leitor
perceber o que foi feito e porque é que é importante/útil. Também
deve permitir que o leitor fique a saber onde encontrar mais
informação sobre o tema do trabalho desenvolvido.

Outras indicações sobre \LaTeX:
\begin{enumerate}
  \item Não se preocupe com as formatações. Utilize as que já estão
    exemplificadas no Capítulo \ref{cap2};
  \item Em especial, não se preocupe com o sítio em que ficam as
    figuras  \textit{flutuantes} (\textit{floating}): estas são
    arrumadas automaticamente pelo \LaTeX;  ficarão tanto melhor
    arrumados quanto mais texto houver; muitos elementos flutuantes
    em pouco texto não costuma permitir um documento com um
    \textit{layout} equilibrado; quando \textbf{concluir} a escrita
    do texto poderá então fazer pequenos ajustes na posição das
    figuras e tabelas utilizando, por exemplo, os seguintes métodos:
    alterar a dimensão das figuras (devem ficar o mais pequenas
    possível desde que legíveis na dimensão A4); inserir quebras de
    página (\verb|\pagebreak|), alterar o local no texto em que surge
    o comando de inserção de cada figura; ou colocar as figuras sem
    serem flutuantes (exemplificado pelo Fig.
      \ref{fig:exemplofigFixa} no Capítulo \ref{cap2}; no entanto
      esta última opção deve ser utilizada com último recurso pois o
      posicionamento automático, desde que com texto suficiente,
      deverá produzir um layout mais equilibrado.
      \item Procure evitar o comando \verb|\pagebreak|. Só o deve
      utilizar no final para eventuais ajustes na paginação.
    \item Deve referir cada uma das figuras, listagens e tabelas
      pelo menos uma vez; utilize sempre o comando
      \verb|\ref{umaLabel}|. Por exemplo: "na Fig.\linebreak
      \verb|\ref{fig:exemplofig}|"\ ou "A Listagem
      \verb|\ref{lst:exemplolst01}|";
    \item Para referir capítulos, secções figuras, listagens e
      tabelas utilize "Capítulo\linebreak \verb|\ref{cap:exemplo}|",
      "Secção \verb|\ref{sec:exemplo}|", "Fig.
      \verb|\ref{fig:exemplo}|", "Listagem
      \verb|\ref{lst:exemplo}|"\ e "Tabela \verb|\ref{tab:exemplo}|",
      respectivamente;
    \item Se pretender forçar uma mudança de linha pode utilizar
      \verb|\\|; se quiser que essa linha partida fique justificada,
      ocupando toda a largura da página, pode utilizar o comando
      \verb|\linebreak|;
    \item Para que o \LaTeX respeite a regra, em português, de hífen
      na mudança de linha, deve utilizar o comando \verb!"-! em lugar
      de \verb!-!. Por exemplo, deve escrever \verb!arco"-íris! em
      lugar de \verb!arco-íris!. Desta forma, quando mudar de linha
      no hífen, a palavra \textbf{arco"-íris} ficará em duas partes:
      "arco-"\ no fim de uma linha e  "-íris"\ no início da linha
      seguinte. Se não conhece esta regra, consulte, por exemplo, a
      seguinte página no Ciberdúvidas:
      \href{https://ciberduvidas.iscte-iul.pt/consultorio/perguntas/a-barra-e-o-hifen-na-translineacao/12731}{https://ciberduvidas.iscte-iul.pt/consultorio/perguntas/a-barra-e-o-hifen-na-translineacao/12731}.
  \end{enumerate}

  «Os minerais são a matéria"-prima essencial para a esmagadora
  maioria dos objetos que utilizamos no nosso quotidiano, desde os
  dispositivos eletrónicos mais complexos aos materiais de construção
  mais básicos. No entanto, a origem e a importância destes recursos
  são frequentemente desconhecidas do público em geral.» Não
  interresa se estás a ler este documento num dispositivo eletrónico,
  ou em impresso, os minerais, eles estão presentes tanto nos
  \textit{microchips} (e não só) assim como no papel também.

  Faz sentido que nunca pensemos na origem dos bens que usamos no
  nosso dia"-a"-dia. Eles aparecem \textit{por magia} nas parteleiras
  das lojas, tu os compras, e os usas para aquilo que tu precisas sem
  pensar muito em todo o processo que foi necessário para esse
  produto existir. Se tu já cozinhaste, mais depressa olhas para uma
  refeição e consegues ter uma ideia de quais foram os ingredientes
  necessários para confécioná"-lo. Bem, imagino que o leitor
  provavelmente nunca fez parte do processo da criação de uma resma
  de papel\footnote{Que também não é concerteza um processo simples e
    linear. É necessário ferramentas e máquinas para cortar as àrvores
    e processar a madeira, e essas ferramentas e máquinas têm que ser
    feitas anteriormente, talvez por outras ferramentas e máquinas, e
    por aí adiante, e essas ferramentas, máquinas e tudo o que é
  necessário durante o processo, usa minerais.}, nem da construção
  das peças de um computador, nem dos processos anteriores a esse
  para as peças das peças existirem, \textbf{incluindo a extração de
    minérios do solo, de onde eles provêm, e por que razão eles são um
  ingrediente necessário} para confecionar o produto final.

  Um painel solar, uma televisão, uma telha ou uma torneira, todos
  estes e muitos mais, normalmente presentes nas próprias casas e
  sendo objetos que usamos no dia"-a"-dia, para o seu fabrico, são
  necessários vários minerais essenciais. Às vezes são coisas que nem
  chegamos a dar o valor que elas realmente possuem, exatamente
  porque o processo de produção está parcial ou completamente
  escondido do consumidor. Não pensamos, por exemplo, na quantidade
  de trabalhadores de várias zonas do globo que fazem o seu trabalho
  às vezes nas condições mais precárias, para extrair o minério da
  natureza usando a sua força humana. Outra vez, o que quer que seja
  que está a tocar o teu pé agora (meia, calçado, o chão da tua casa,
  etc., etc.), por exemplo, não apareceu magicamente na tua vida
  porque alguém teve dinheiro para o comprar, existe toda uma cadeia de
  trabalho humano envolvida.

  \section{Objetivos}

  \begin{itemize}
    \item Promover a literacia científica sobre os minerais e a sua
      importância no quotidiano;
    \item Usar a \textbf{gamificação} como uma ferramenta pedagógica
      para tornar «a aprendizagem em experiências interativas e
      motivadoras», mais envolventes e visualmente apelativas — a
      criação de um jogo educativo e recreativo.
  \end{itemize}

  % No jogo, através da exploração de uma casa virtual, o jogador
  % poderá clicar em diferentes objetos do dia"-a"-dia como uma
  % televisão, um frigorifico ou uma torneira, e descobrir os minerais
  % essenciais para o seu fabrico, a sua proveniência e relevância.

  % deve ficar claro porque é que é importante/útil resolver esse problema

  \section{Estrutura do documento}

  \begin{enumerate}
    \item ...
  \end{enumerate}

  \chapter{Desenvolvimento}
  \label{develop}

  \section{Enquadramento teórico}

  \subsection{"Estado da arte"}

  «Inspirado por jogos educativos como
  "\href{https://www.thebriefcasegame.eu/}{The Briefcase Game}",
  [...] o objetivo é aumentar a literacia sobre recursos
  geológicos, demonstrando de forma lúdica e visual a ligação direta
  entre a geologia e a nossa vida diária.»

  % O nosso toque pessoal, point click tipo Club Penguin...

  \subsection{Os minerais}

  Eles andam por aí espalhados, mas tu nem pensas neles: os minerais.
  Matéria-prima e recurso importante para diversos setores como a
  indústria química, da construção civil, da agricultura e da
  energia. Mas é que estão presentes em todo o lado mesmo: se estás a
  ler isto através de um aparelho eletrónico, esquece, desde o cobre
  dos cabos elétricos até ao silício dos microchips, mas mesmo lendo
  numa folha de papel, é capaz dessa folha ter vestígios de cádmio,
  cobalto e titânio. As frutas e vegetais que consomes, além das
  vitaminas, também têm minerais. Minerais são indispensáveis para a
  nossa vida e para a sociedade.

  E esse seria o objetivo principal deste jogo, ensinar onde é que
  estes minerais, que existem naturalmente no nosso planeta, estão
  presentes nas coisas que usamos todos os dias. Também podemos
  ensinar talvez qual a utilidade de certo mineral numa ocasião
  específica. Mas basta isso?

  Uma das razões para a qual estes minerais ficam escondidos dos
  consumidores é porque, tu vais à loja e compras o telemóvel, mas o
  processo de produção desse telemóvel, da extração dos minerais da
  terra, da montagem do equipamento, até ao produto final, é
  totalmente ofuscado. Existe uma alienação entre as nossas vidas e a
  geologia, os minerais, e todo o trabalho humano que leva à
  realização dos teus bens. Simplesmente o telemóvel não apareceu na
  estante da loja por pozinhos de perlimpimpim. O telemóvel é visto
  como uma coisa e não como fruto de mineração e do trabalho.

  \subsection{Gamificação na pedagogia}

  A "Casa dos Minerais" utilizará elementos de gamificação para
  estruturar a aprendizagem, como a progressão por diferentes níveis
  e a inclusão de minijogos, por exemplo, um questionário no final de
  cada nível.

  No entanto, a abordagem pedagógica deste projeto foca-se em usar a
  brincadeira como uma forma de descoberta e reflexão. O objetivo
  principal não é criar um sistema de avaliação rigoroso; o foco é a
  alegria de aprender. A aprendizagem é mais importante do que uma
  pontuação final ou a rapidez com que se termina o jogo.

  Desta forma, elementos como a pontuação ou um temporizador,
  servirão como um incentivo opcional (conforme detalhado na secção
  \ref{future}) e não como o foco central da experiência.

  \subsection{Para quem é o jogo?}
  \label{forwho}

  O jogo têm como alvo principal as crianças e jovens, e serviria de
  apoio para o que eles estão a aprender nas aulas de Estudo de Meio
  e Ciências Naturais (no contexto de Portugal). Mas a verdade é que
  todos merecem aprender, por isso precisamos de pensar em fazer um
  jogo para todas as faixas etárias. Desde que tenha o interesse em
  adquirir conhecimento acerca dos minerais.

  Mas esta inclusão não pode parar na idade. Existem pessoas com
  limitações a nível físico (controlo, mobilidade), cognitivo
  (pensamento, memória, processamento de informação), sensorial,
  visão, audição e/ou fala, e elas também merecem de ter o direito à
  sabedoria (e à diversão de jogar).

  O que pode parecer trabalhoso ter de se preocupar com todos na
  realidade pode ser algo bem simples de alcançar, por exemplo,
  escolhendo a tipografia correta, baseado no contraste da cor de
  letra, do tamanho de letra, e no tipo de letra. Especialmente no
  ambiente de navegadores web, estes estão muito preparados para
  serem úteis e utilizáveis por todos, basta aos programadores
  fazerem o seu trabalho corretamente.

  \section{Metodologia}

  Para a implementação, o projeto baseia-se em tecnologias de
  multiplataforma, tirando partido da linguagem de programação Odin.
  Para a parte gráfica escolhemos a Raylib, integrada com Odin. Esta
  combinação permite criar gráficos 2D interativos com desempenho
  nativo e exportação para WebAssembly (WASM), garantindo execução
  fluida diretamente no navegador.

  \subsection{Como integrar outros públicos?}

  Os problemas falados em \ref{forwho}, já foram todos pensados, e já
  existem soluções para alguns dos problemas. Nós só necessitamos
  implementar essas soluções.

  \subsubsection{Acessibilidade}

  No que toca a Web, existe a \href{https://www.w3.org/WAI/}{Web
  Accessibility Initiative (WAI)} do \href{https://www.w3.org/}{World
  Wide Web Consortium (W3C)}. Criaram um conjunto normas para criar
  um Web acessível, e nos só temos de seguir as normas: HTML
  semântico e ARIA; texto para leitores de ecrã; texto simples;
  imagens e vídeos (média) acessíveis; controlos de interface
  semânticos; cores e contraste; ...

  \subsubsection{Internacionalização}

  Aprender não esta reservado para uma só cultura, região, língua,
  etnia... A W3C também esta preocupada neste aspeto, e vemos isso
  por exemplo com a
  \href{https://developer.mozilla.org/en-US/docs/Web/JavaScript/Reference/Global_Objects/Intl}{Intl
  Web API}. Não somos poliglotas, mas com certeza encontraremos forma
  de permitir com que este projeto não seja apenas em português e
  inglês. Os browsers normalmente conseguem traduzir um website, mas
  se não quisermos depender disso, podemos explorar opções como
  \href{https://github.com/argosopentech/argos-translate}{Argos}.

  \subsubsection{Informação semântica}

  O jogo quando jogado no navegador, não podemos depender só do HTML
  `canvas` para mostrar a informação, porque a informação no `canvas`
  não passa de pixéis, de bits num bitmap (um leitor de ecrãs não
  consegue ler texto no `canvas`). Devemos usar HTML semântico para
  qualquer informação que seja de valor.

  \subsubsection{Independência de uma plataforma}

  O jogo tem de ser possível de ser jogado na maioria dos
  dispositivos, e não só no nosso computador. Existem pessoas que
  talvez apenas tenham um smartphone, e que nem é dos “bons”. Temos
  dois problemas aqui então. Para o primeiro, a solução vai ser não
  criar o jogo apenas para navegador web através da biblioteca
  \href{https://www.raylib.com/index.html}{raylib}, e então o jogo
  poderá rodar nativamente num PC (caso o PC não tenha suporte para
  os browsers mais atualizados) independentemente do sistema
  operativo, num Android, numa PlayStation da Sony ou numa Switch da
  Nintendo. Para o segundo problema, a performance: tentar escrever
  código otimizado, a pensar que existem computadores mais fraquitos;
  controlar a quantidade e o tamanho das requisições, a pensar nos
  sítios onde a internet e mais lenta; caches; armazenamento local; o
  offline. O jogo não vai ser só para os computadores mais topo de
  gama. Não vais ter de comprar um novo dispositivo para jogar. Uma
  decisão ecológica e amiga do ambiente, mas também amiga da tua carteira!
