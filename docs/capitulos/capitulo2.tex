\chapter{Desenvolvimento}
\label{ch:develop}

\section{Enquadramento teórico}

\subsection{``Estado da arte''}

«Inspirado por jogos educativos como \textcite{briefcasegame},
[...] o objetivo é aumentar a literacia sobre recursos
geológicos, demonstrando de forma lúdica e visual a ligação direta
entre a geologia e a nossa vida diária.»

% O nosso toque pessoal, point click tipo Club Penguin...

\subsection{Os minerais}
\label{sec:minerals}

Eles andam por aí espalhados, mas tu nem pensas neles: os minerais.
Matéria-prima e recurso importante para diversos setores como a
indústria química, da construção civil, da agricultura e da
energia. Mas é que estão presentes em todo o lado mesmo:

\begin{itemize}
  \item Se estás a ler isto através de um aparelho eletrónico,
    esquece, desde o cobre e ouro dos cabos elétricos até ao
    silício dos microchips;
  \item mas mesmo lendo numa folha de papel, é capaz dessa folha
    ter vestígios de cádmio, cobalto e titânio.
\end{itemize}

As frutas e vegetais que consomes, além das vitaminas, também têm os
minerais presentes no solo. Minerais são indispensáveis para a
nossa vida e para a sociedade.

O objetivo principal deste jogo é ensinar onde é que estes
minerais, que existem naturalmente no nosso planeta, estão
presentes nas coisas que usamos todos os dias. Além isso,
ensinamos qual a utilidade de certo mineral numa ocasião específica.
Mas basta isso?

Uma das razões para a qual estes minerais ficam escondidos dos
consumidores é porque, tu vais à loja e compras o telemóvel, mas o
processo de produção desse telemóvel, da extração dos minerais da
terra, da montagem do equipamento, até ao produto final, é
totalmente ofuscado. Existe uma alienação entre as nossas vidas e a
geologia — os minerais — e todo o trabalho humano que leva à
realização dos teus bens. Simplesmente o telemóvel não apareceu na
estante da loja por pozinhos de perlimpimpim. O telemóvel é visto
como uma coisa e não como fruto de mineração e do trabalho. Tudo
isto já havia sido dito na \ref{ch:intro}.

Usamos o jogo para conectar também o mineral, ao país de origem
(geografia), e ao trabalho e/ou processo necessário para esse
material estar disponível.

\subsection{Gamificação na pedagogia}

A ``Casa dos Minerais'' utilizará elementos de gamificação para
estruturar a aprendizagem, como a progressão por diferentes níveis
e a inclusão de minijogos, por exemplo, um questionário no final de
cada nível.

No entanto, a abordagem pedagógica deste projeto foca-se em usar a
brincadeira como uma forma de descoberta e reflexão. O objetivo
principal não é criar um sistema de avaliação rigoroso; o foco é a
alegria de aprender. A aprendizagem é mais importante do que uma
pontuação final ou a rapidez com que se termina o jogo.

Desta forma, elementos como a pontuação ou um temporizador,
servirão como um incentivo opcional (conforme detalhado na secção
\ref{sec:future}) e não como o foco central da experiência.

\subsection{Para quem é o jogo?}
\label{sec:forwho}

O jogo têm como alvo principal as crianças e jovens, e serviria de
apoio para o que eles estão a aprender nas aulas de Estudo de Meio
e Ciências Naturais (no contexto de Portugal). Mas a verdade é que
todos merecem aprender, por isso precisamos de pensar em fazer um
jogo para todas as faixas etárias. Desde que tenha o interesse em
adquirir conhecimento acerca dos minerais.

Mas esta inclusão não pode parar na idade. Existem pessoas com
limitações a nível físico (controlo, mobilidade), cognitivo
(pensamento, memória, processamento de informação), sensorial,
visão, audição e/ou fala, e elas também merecem de ter o direito à
sabedoria (e à diversão de jogar).

O que pode parecer trabalhoso ter de se preocupar com todos na
realidade pode ser algo bem simples de alcançar, por exemplo,
escolhendo a tipografia correta, baseado no contraste da cor de
letra, do tamanho de letra, e no tipo de letra. Especialmente no
ambiente de navegadores web, estes estão muito preparados para
serem úteis e utilizáveis por todos, basta aos programadores
fazerem o seu trabalho corretamente.

\section{Metodologia}

Para a implementação, o projeto baseia-se em tecnologias de
multiplataforma, tirando partido da linguagem de programação Odin.
Para a parte gráfica escolhemos a Raylib, integrada com Odin. Esta
combinação permite criar gráficos 2D interativos com desempenho
nativo e exportação para WebAssembly (WASM), garantindo execução
fluida diretamente no navegador.

\subsection{Identificação dos minerais usados no quotidiano}

Minerais são todas as substâncias naturais formadas por processos
geológicos que normalmente consegues encontrar na crosta da terra.

A telhado das casas podem ser feitos com argila, os vidros com areia.
Na cozinha, tens pratos que podem conter areia, calcário, feldspato e
argila. O sal vem do sal-gema. Uma panela pode ser de ferro, níquel e
crómio. O frigorifico pode ser alumínio e cobre, e derivados de
petróleo. Na cozinha também pode ser usado gás natural. Existem
fotografias que usam prata e bronze. Um vaso de flores pode ser de
argila. Um livro pode ter vestígios de cadmio, cobalto e titânio. Uma
lâmpada pode ter níquel, alumínio e chumbo. Um relógio pode ter
silício, crómio, ferro e níquel. Um espelho terá quartzo, alumínio e
prata. O papel higiénico gesso e calcário. A pasta de dentes tem fluorite.

O mais impressionante talvez são os bens tecnológicos como
televisões/monitores, computadores e telemóveis, tão presentes no
nosso dia"-a"-dia e que os levamos de um lado para o outro connosco.
Estes podem conter zinco, crómio, silício, alumínio, cobre, estanho,
manganês, cobalto, bário, platina, tungsténio, arsénio, lítio, ouro, prata, ...

\subsection{Comparação com outros jogos}

Em comparação com o \textcite{briefcasegame}, o nosso jogo seria um
“mundo” aberto, onde o jogador começava dentro de uma casa e teria de
encontrar os objetos, para ver os minerais que contém neles. Enquanto
no BriefCase game, o jogador só precisa de associar o objeto ao mineral.

No \textcite{min4kids-minerals-house}, o objetivo é parecido ao jogo
anterior só que neste caso, não tem aspeto de “gaming” é mais
informacional, oque o nosso jogo também tem.

No \textcite{wordwall-rochas-minerais}, é um conjunto de jogos sobre
minerais, mas muito simples em comparação ao nosso.

\subsection{Conceito, regras e âmbito do jogo}

O conceito central do ``Casa dos Minerais'' assenta na exploração
interativa e autoguiada de um ambiente virtual, primariamente uma
casa. Inspirado num formato point-and-click, semelhante ao ``Club
Penguin'', o jogador terá a liberdade de navegar por diferentes
divisões, que funcionarão como níveis.

Ao interagir com os objetos do quotidiano (como um frigorífico, uma
  televisão ou uma
torneira), o jogador descobre informações sobre os minerais
essenciais para o seu fabrico, a
sua proveniência e a sua relevância.

Embora a exploração dentro de cada nível seja livre, permitindo ao
jogador criar o seu próprio
percurso de aprendizagem, a progressão no jogo será linear e estruturada

\subsubsection{Regras e Progressão}

\begin{itemize}
  \item Estrutura de Níveis: O jogo será dividido em níveis (ex:
    cozinha, sala, exterior da casa).
  \item Desbloqueio de Níveis: A regra principal de progressão é que
    o jogador só poderá avançar para o nível seguinte após ter
    completado com sucesso o nível anterior.
  \item Validação da Aprendizagem: A conclusão de um nível será
    validada através de um minijogo, especificamente um questionário
    sobre os minerais e informações exploradas nessa área. A
    aprovação neste questionário é o que permite o desbloqueio do próximo nível.
  \item Sistema de Pontuação (Opcional): Está a ser ponderada a
    inclusão de um sistema de pontuação, que poderá ser influenciado
    pelo desempenho no questionário. Uma ideia em discussão é a
    inclusão de um temporizador por nível; um tempo de conclusão mais
    rápido poderia resultar numa pontuação mais alta, incentivando a
    rejogabilidade.
\end{itemize}

\subsubsection{Âmbito do Jogo}

O âmbito do jogo foca-se na experiência de descoberta dentro deste
``mundo'' virtual. O jogador não tem ``regras'' de exploração; pode
clicar em qualquer objeto, em qualquer ordem, dentro do nível ativo.
A ``Casa dos Minerais'' serve, assim, como o hub central para esta descoberta.

\subsection{Como integrar outros públicos?}
\label{sec:other}

Os problemas falados em \ref{sec:forwho}, já foram todos pensados, e já
existem soluções para alguns dos problemas. Nós só necessitamos
implementar essas soluções.

\subsubsection{Acessibilidade}

No que toca a Web, existe a \href{https://www.w3.org/WAI/}{Web
Accessibility Initiative (WAI)} do \href{https://www.w3.org/}{World
Wide Web Consortium (W3C)}. Criaram um conjunto normas para criar
um Web acessível, e nos só temos de seguir as normas: HTML
semântico e ARIA; texto para leitores de ecrã; texto simples;
imagens e vídeos (média) acessíveis; controlos de interface
semânticos; cores e contraste; ...

\subsubsection{Internacionalização}

Aprender não esta reservado para uma só cultura, região, língua,
etnia... A W3C também esta preocupada neste aspeto, e vemos isso
por exemplo com a
\href{https://developer.mozilla.org/en-US/docs/Web/JavaScript/Reference/Global_Objects/Intl}{Intl
Web API}. Não somos poliglotas, mas com certeza encontraremos forma
de permitir com que este projeto não seja apenas em português e
inglês. Os browsers normalmente conseguem traduzir um website, mas
se não quisermos depender disso, podemos explorar opções como
\href{https://github.com/argosopentech/argos-translate}{Argos}.

\subsubsection{Informação semântica}

O jogo quando jogado no navegador, não podemos depender só do HTML
`canvas` para mostrar a informação, porque a informação no `canvas`
não passa de pixéis, de bits num bitmap (um leitor de ecrãs não
consegue ler texto no `canvas`). Devemos usar HTML semântico para
qualquer informação que seja de valor.

\subsubsection{Independência de uma plataforma}

O jogo tem de ser possível de ser jogado na maioria dos
dispositivos, e não só no nosso computador. Existem pessoas que
talvez apenas tenham um smartphone, e que nem é dos “bons”. Temos
dois problemas aqui então. Para o primeiro, a solução vai ser não
criar o jogo apenas para navegador web através da biblioteca
\href{https://www.raylib.com/index.html}{raylib}, e então o jogo
poderá rodar nativamente num PC (caso o PC não tenha suporte para
os browsers mais atualizados) independentemente do sistema
operativo, num Android, numa PlayStation da Sony ou numa Switch da
Nintendo. Para o segundo problema, a performance: tentar escrever
código otimizado, a pensar que existem computadores mais fraquitos;
controlar a quantidade e o tamanho das requisições, a pensar nos
sítios onde a internet e mais lenta; caches; armazenamento local; o
offline. O jogo não vai ser só para os computadores mais topo de
gama. Não vais ter de comprar um novo dispositivo para jogar. Uma
decisão ecológica e amiga do ambiente, mas também amiga da tua carteira!

\section{\textit{Tech Stack}}

Começamos por pensar de que forma íamos implementar o jogo. Como o
nosso objetivo e eventualmente o jogo ser multiplataforma, surgiu a
ideia de usarmos a biblioteca \href{https://www.raylib.com/}{Raylib}.
E uma biblioteca bem simples, mas que faz tudo o que necessitamos. E
fácil de usar e tem uma boa documentação e exemplos. Agora só tivemos
de pensar como e que faríamos para usar o Raylib para a Web. A
resposta talvez seja obvia, precisávamos de usar WebAssembly (WASM).
Mas ainda tivemos de decidir entre compilar a biblioteca Raylib para
WASM, ou ter a logica do jogo em WASM mas depois usar as API que os
navegadores Web oferecem para criar a parte gráfica do jogo.
Escolhemos a segunda. Parece ser a que causa menos problemas, apesar
de que agora temos de manter duas interfaces gráficas diferentes: A
que usa a API Canvas do HTML, e a que usa Raylib.

Bom, e agora, em que linguagem vamos criar a logica e a interface
gráfica do jogo? Tínhamos basicamente qualquer linguagem com FFI,
quase todas as linguagens compiladas e que conseguem dar link com
binarios compilados criando ABI shims. Raylib e implementada em C,
mas a comunidade já criou bindings para múltiplas linguagens. Nós
queríamos usar uma linguagem que ainda não tínhamos usado, como forma
de aprender essa linguagem e os seus paradigmas, mas também para
mostrar que somos programadores, independentemente da linguagem. Um
de nos lembrou-se da linguagem \href{https://odin-lang.org/}{Odin} e
não discutimos mais acerca disso. Por acaso a linguagem já tem até um
binding oficial para o Raylib, nesse sentido facilitou o nosso
trabalho. Odin oferece programação de vetores, matrizes e quaterniões
como tipo de dados, e estruturas de dados que facilitam programação
SIMD e SIMT, que talvez não iremos usar diretamente, mas que com
certeza ajuda bastante na criação de jogos 3D, na performance.

Também tivemos a surpresa da linguagem estar relativamente preparada
para compilação em WebAssembly, sem ter de fazer trabalho
intermediário com
\href{https://emscripten.org/index.html}{Emscripten}, LLVM e
ficheiros de objetos, e ainda a gestão da memoria no JavaScript, já
que Odin oferece um ficheiro preparado para o runtime, com integração
com DOM e até WebGL.

Odin não é uma linguagem qualquer, e já tem usos no mundo
profissional, por exemplo, \href{https://jangafx.com/}{JangaFX}
desenvolve ferramentas como EmberGen, GeoGen e LiquiGen inteiramente
em Odin, mostrando o desempenho e a facilidade da linguagem para
aplicações gráficas complexas. E é isso. Depois conforme a
necessidade podemos ir buscar outras bibliotecas, mas por enquanto, basta.

%%%

% \begin{figure}[!htb]
%   \centering
%   \caption{Exemplo de Figura flutuante (\textit{in}
%   \url{http://en.wikipedia.org/wiki/Class_diagram})}
%   \includegraphics[width=13cm]{exemploFig}
%   \label{fig:exemplofig}
% \end{figure}
%
% \begin{center}
%
%   \captionof{figure}{Exemplo de Figura não flutuante (\textit{in}
%   \url{http://en.wikipedia.org/wiki/Class_diagram})}
%   \includegraphics[width=\textwidth]{exemploFig}
%   %\includegraphics[width=0.5\textwidth]{exemploFig}
%   \label{fig:exemplofigFixa}
% \end{center}

% \lstinputlisting[language=java,abovecaptionskip=20pt,caption={Exemplo
%     de listagem de parte de um ficheiro Java na pasta
% listagens.},{label=lst:javaparcial},firstline=38,lastline=44]{listagens/quicksort.java}

% \noindent
% \begin{minipage}{\linewidth}
%   \captionof{table}{Uma tabela de exemplo não flutuante} % título da tabela
%   \begin{tabular}{l l c r} % duas colunas à esquerda (l l), uma ao
%     % centro (c) e uma à direita (r) (4 colunas)
%     \hline\hline %insere duas linhas horizontais
%     Caso & Método 1 & Método 2 & Método 3 \\ [0.5ex] % insere tabela
%     \hline % insere uma linha horizontal
%     1 & 50 & 837 & 970 \\ % insure corpo da tabela
%     2 & 47 & 877 & 230 \\
%     3 & 31 & 25 & 415 \\
%     4 & 35 & 144 & 2356 \\
%     5 & 45 & 300 & 556 \\ [1ex] % [1ex] adiciona espaço vertical
%     \hline %insere uma linha horizontal
%   \end{tabular}
%   \label{tab:casoFIXO} % é utilizada para referir a tabela no texto
%   \\ %% linha em braco
%   \textit{Nota}: uma nota. remover se não for necessária.\\
% \end{minipage}
%
% %baseado num exemplo em http://www1.maths.leeds.ac.uk/latex/TableHelp1.pdf
% \begin{table}[!htb]
%   \caption{Uma tabela de exemplo}
%   \begin{tabular}{l l c r} % duas colunas à esquerda (l l), uma ao
%     % centro (c) e uma à direita (r) (4 colunas)
%     \hline\hline %insere duas linhas horizontais
%     Caso & Método 1 & Método 2 & Método 3 \\ [0.5ex] % insere tabela
%     \hline % insere uma linha horizontal
%     1 & 50 & 837 & 970 \\ % insere corpo da tabela
%     2 & 47 & 877 & 230 \\
%     3 & 31 & 25 & 415 \\
%     4 & 35 & 144 & 2356 \\
%     5 & 45 & 300 & 556 \\ [1ex] % [1ex] adiciona espaço vertical
%     \hline %insere uma linha horizontal
%   \end{tabular}
%   \label{tab:caso} % é utilizada para referir a tabela no texto
%   \\ %% linha em braco
%   \textit{Nota}: uma nota. remover se não for necessária.\\
% \end{table}
