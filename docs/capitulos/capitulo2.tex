\chapter{Desenvolvimento}
\label{ch:develop}

\section{Enquadramento teórico}

\subsection{Estado da arte}

Existem já alguns jogos e recursos digitais sobre minerais e
geociências. Para perceber onde a \textit{Casa dos Minerais} se
encaixa, analisámos o que já existe e o que falta.

Comparativamente ao \textcite{briefcasegame}, a \textit{Casa dos
Minerais} aposta numa abordagem mais exploratória. Enquanto o
\textit{BriefCase} foca na associação imediata objeto–mineral, o
nosso projeto privilegia a contextualização espacial (o objeto no seu
``habitat'') antes da validação. Relativamente ao
\textcite{min4kids-minerals-house}, que funciona mais como uma
enciclopédia interativa, o nosso jogo introduz um ciclo de
\textit{gameplay} com desafios (quizzes) para testar a retenção. Por
fim, ao contrário dos exercícios isolados do
\textcite{wordwall-rochas-minerais}, oferecemos um cenário
persistente e contínuo.

A Tabela~\ref{tab:comparacao-jogos} resume estas diferenças. O ponto
que mais diferencia o nosso jogo é a combinação de exploração livre
com questionário de validação.

\begin{longtable}{p{3.0cm} p{3.5cm} p{4.0cm} p{3.5cm}}
  \caption{Comparação entre recursos/jogos educativos sobre
  minerais.} \label{tab:comparacao-jogos} \\
  \hline
  \textbf{Recurso} & \textbf{Tipo de interação} & \textbf{Contexto
  (cenário)} & \textbf{Avaliação} \\
  \hline
  \endfirsthead
  \multicolumn{4}{c}%
  {\tablename\ \thetable\ -- \textit{Continuação da página anterior}} \\
  \hline
  \textbf{Recurso} & \textbf{Tipo de interação} & \textbf{Contexto
  (cenário)} & \textbf{Avaliação} \\
  \hline
  \endhead
  \hline \multicolumn{4}{r}{\textit{Continua na próxima página}} \\
  \endfoot
  \hline
  \endlastfoot

  \textcite{briefcasegame}
  & Associação direta objeto–mineral
  & Pouco contexto espacial; foco na correspondência
  & Feedback imediato da associação \\

  \textcite{min4kids-minerals-house}
  & Consulta/informação
  & Conteúdo mais informacional
  & Sem validação estruturada \\

  \textcite{wordwall-rochas-minerais}
  & Mini"-jogos simples
  & Contexto reduzido; tarefas isoladas
  & Feedback rápido por exercício \\

  \textit{Casa dos Minerais}
  & Exploração \textit{point"-and"-click} + Questionário
  & Cenários familiares (casa/escola)
  & Quiz curto para reforço e desbloqueio de nível \\
\end{longtable}

\subsection{Os minerais}
\label{sec:minerals}

Eles andam por aí espalhados, mas tu nem pensas neles: os minerais.
Matéria-prima e recurso importante para diversos setores como a
indústria química, da construção civil, da agricultura e da
energia. Mas é que estão presentes em todo o lado mesmo:

\begin{itemize}
  \item Se estás a ler isto através de um aparelho eletrónico,
    esquece, desde o cobre e ouro dos cabos elétricos até ao
    silício dos microchips;
  \item Mas mesmo lendo numa folha de papel, é capaz dessa folha
    ter vestígios de cádmio, cobalto e titânio.
\end{itemize}

As frutas e vegetais que consomes, além das vitaminas, também têm os
minerais presentes no solo. Minerais são indispensáveis para a
nossa vida e para a sociedade.

O objetivo principal deste jogo é ensinar onde é que estes
minerais, que existem naturalmente no nosso planeta, estão
presentes nas coisas que usamos todos os dias. Além isso,
ensinamos qual a utilidade de certo mineral numa ocasião específica.
Mas basta isso?

Uma das razões para a qual estes minerais ficam escondidos dos
consumidores é porque, tu vais à loja e compras o telemóvel, mas o
processo de produção desse telemóvel, da extração dos minerais da
terra, da montagem do equipamento, até ao produto final, é
totalmente ofuscado. Existe uma alienação entre as nossas vidas e a
geologia — os minerais — e todo o trabalho humano que leva à
realização dos teus bens. Simplesmente o telemóvel não apareceu na
estante da loja por pozinhos de perlimpimpim. O telemóvel é visto
como uma coisa e não como fruto de mineração e do trabalho. Tudo
isto já havia sido dito na \ref{ch:intro}.

Usamos o jogo para conectar também o mineral, ao país de origem
(geografia), e ao trabalho e/ou processo necessário para esse
material estar disponível.

\subsection{Gamificação na pedagogia}

A ``Casa dos Minerais'' usa elementos de gamificação para dar
estrutura à aprendizagem. A mecânica principal é um
\textbf{questionário no final de cada nível} (ver Secção~\ref{sec:questions}).

Na prática, o questionário funciona como um ``guardião'' ou
\textit{boss fight} educativa: o jogador explora livremente, e no
final tem de demonstrar o que aprendeu para desbloquear o nível
seguinte. O conhecimento é, literalmente, a chave para avançar.

A ideia não é punir quem erra, mas incentivar a descoberta. Por isso,
elementos mais competitivos como pontuações ou temporizadores ficaram
de fora desta versão, podendo ser introduzidos mais tarde (como dito
na Secção~\ref{sec:future}).

\subsection{Para quem é o jogo?}
\label{sec:forwho}

O público-alvo principal são crianças e jovens em idade escolar, como
apoio a Estudo do Meio e Ciências Naturais. Dito isto, o visual e a
mecânica de exploração foram pensados para funcionar com qualquer faixa etária.

Do lado da acessibilidade, usámos o tipo de letra OpenDyslexic \autocite{OpenDyslexic}, criada para facilitar a leitura para pessoas com deslexia. Caso o uso da mesma não seja possível, como \textit{fallback} usamos uma \textbf{tipografia sem serifa (sans-serif)}. Nas duas ocasiões, é utilizado um espaçamento generoso e bom contraste, pensando em utilizadores com dificuldades de leitura. O protótipo atual ainda não suporta navegação total por teclado nem leitores de ecrã (ver Secção~\ref{sec:limitacoes}), mas estas escolhas visuais já apontam nessa direção.

\subsection{Aprendizagem por descoberta e contextualização}

A ideia por trás da aprendizagem por descoberta é simples: quem
participa ativamente compreende melhor. Na \textit{Casa dos
Minerais}, isto traduz-se em ancorar conceitos abstratos (como
``recurso mineral'') em exemplos concretos do dia a dia.

\subsubsection{Ancoragem na Cadeia Produtiva}
Conceitos como ``Recurso Mineral'' ou ``Extração'' podem parecer
distantes. O jogo encurta essa distância ao ligar o objeto familiar
(ex: a torradeira na cozinha) à sua origem geológica.

Esta estrutura permite, teoricamente, criar níveis para cada etapa da
cadeia de produção:
\begin{enumerate}
  \item \textbf{Nível Casa:} Onde usamos o produto final.
  \item \textbf{Nível Fábrica:} Onde vemos a montagem dos componentes.
  \item \textbf{Nível Mina:} Onde ocorre a extração do minério bruto.
\end{enumerate}

Deste modo, o jogador pode seguir o ``caminho inverso'' da produção
dos seus bens, compreendendo que o telemóvel não nasce na loja, mas
começa como rocha na terra. O sistema de níveis do jogo foi desenhado
para suportar essa expansão temática, permitindo adicionar novos
cenários que representem estas diferentes etapas.

\subsection{Ciclo pedagógico e questionários como reforço}

O ciclo do jogo tem duas fases. Primeiro, o utilizador explora o
cenário e vai absorvendo os conceitos no seu contexto. Depois,
responde a um questionário curto que o obriga a recuperar ativamente
o que observou.

\subsection{Qualidade do conteúdo e fontes}

Para que o conteúdo do jogo seja fiável e útil, seguiram-se algumas práticas:

\begin{itemize}
  \item \textbf{Fontes fiáveis:} As associações foram baseadas em
    fontes reconhecidas como a USGS
    \textcite{usgs_education_mrp,usgs_minerals101} e a Minerals
    Education Coalition \textcite{mec_mineralsinyourlife}.
  \item \textbf{Coerência e Estrutura:} Utilização consistente de
    terminologia e estrutura uniforme nas descrições (Mineral, Uso, Origem).
  \item \textbf{Linguagem adequada:} Textos claros e diretos.
\end{itemize}

\section{Metodologia}

O projeto usa a linguagem de programação Odin e a biblioteca gráfica
Raylib. O Odin compila para WebAssembly (WASM), o que permite correr
o jogo no navegador com desempenho próximo do nativo. No browser, o
módulo WASM é integrado com HTML5, CSS3 e JavaScript para tratar da
parte visual e da interação.

\subsection{Identificação dos minerais usados no quotidiano}

Minerais são todas as substâncias naturais formadas por processos
geológicos que normalmente consegues encontrar na crosta da terra.
A telha das casas pode ser feita com argila, o vidro com areia (quartzo).
Na cozinha, tens pratos que podem conter feldspato, e panelas de ferro e crómio.
O mais impressionante são os bens tecnológicos como telemóveis, que
contêm uma vasta gama de metais raros e preciosos como ouro, lítio e cobalto.

\subsection{Critérios de seleção de objetos}

A seleção de objetos para os níveis ``casa'' e ``escola'' foi guiada
por critérios pedagógicos e práticos: objetos comuns, facilmente
reconhecíveis e com associações minerais claras e educativas.

\subsection{Diferenças de abordagem (exploração vs associação direta)}

Há duas formas comuns de apresentar este tipo de conteúdo: na
\textbf{exploração}, o utilizador aprende ao interagir com um
ambiente; na \textbf{associação direta}, o foco é a memorização rápida de pares.

O \textit{Casa dos Minerais} mistura as duas: primeiro explora-se
para ganhar contexto, depois faz-se um quiz para verificar e treinar.

\subsection{Limitações observadas e decisões adotadas no projeto}

Ao analisar outros recursos e ao longo do desenvolvimento,
encontrámos dois problemas recorrentes. O primeiro é a tendência para
conteúdos demasiado expositivos , o utilizador lê informação mas não
tem oportunidade de a aplicar. O segundo é a associação direta sem
contexto: decorar que ``o cobre está no fio elétrico'' não ajuda
muito se não se perceber porquê.

Para contornar isso, o jogo organiza a aprendizagem em dois momentos:
primeiro a exploração num cenário familiar, onde a informação aparece ligada a
objetos do dia a dia; depois um questionário curto de escolha múltipla para
reforço e validação.

Do ponto de vista visual, houve limitações na produção de assets.
Testámos a geração de elementos gráficos com o Gemini,
mas os resultados não serviam para um mapa baseado em tiles: faltava uma
grelha de \textit{tileset} 32$\times$32\,px. As imagens geradas vinham em
dimensões grandes demais e sem organização reutilizável, o que tornava difícil
o recorte, alinhamento e integração no cenário.
Por isso, comprámos um conjunto de assets já estruturados,
compatíveis com o formato do mapa e visualmente consistentes.

% todo meter melhores imagens com tamanhos parecidos para nao ficar aquele bloco
\begin{figure}[H]
  \centering
  \begin{subfigure}[t]{0.58\textwidth}
    \centering
    \includegraphics[width=\textwidth]{figuras/gemini_assets.jpeg}
    \caption{Asset gerado automaticamente (Gemini): imagem
      visualmente apelativa,
      mas sem estrutura de \textit{tileset} e sem grelha 32$\times$32\,px,
    dificultando recorte, alinhamento e reutilização no mapa.}
    \label{fig:gemini-gerado}
  \end{subfigure}
  \hfill
  \begin{subfigure}[t]{0.38\textwidth}
    \centering
    \includegraphics[width=\textwidth]{figuras/paid_assets.png}
    \caption{Tileset estruturado: elementos organizados e dimensionados,
    facilitando integração, consistência visual e reutilização no cenário.}
    \label{fig:tileset-estruturado}
  \end{subfigure}

  \caption{Comparação entre geração automática e assets estruturados
    para jogos 2D.
    Apesar da qualidade estética, a ausência de formato e escala
    (ex.: 32$\times$32\,px)
  na geração automática limitou a utilização prática no projeto.}
  \label{fig:comparacao-gemini-tileset}
\end{figure}

A Figura~\ref{fig:comparacao-gemini-tileset} ilustra que a principal limitação
não foi estética, mas sim técnica: o projeto exigia assets reutilizáveis em
tilesets com dimensões uniformes (32$\times$32\,px) para montagem
eficiente do mapa.

Em contrapartida, a utilização de IA (Gemini) revelou-se bastante
útil para a criação de elementos individuais, nomeadamente as
representações em \textit{pixel art} dos próprios minerais. Ao
contrário dos \textit{tilesets} de mapa que exigem encaixe perfeito,
as imagens isoladas foram geradas com qualidade satisfatória,
necessitando apenas de pequenas correções pontuais (como limpeza de
fundo ou ajustes de detalhes) realizadas em software de edição de
imagem como o Photopea ou Photoshop.

\begin{figure}[H]
  \centering
  \begin{subfigure}[t]{0.3\textwidth}
    \centering
    \frame{\includegraphics[width=0.6\textwidth]{figuras/mineral_ouro.png}}
    \caption{Ouro}
  \end{subfigure}
  \hfill
  \begin{subfigure}[t]{0.3\textwidth}
    \centering
    \frame{\includegraphics[width=0.6\textwidth]{figuras/mineral_litio.png}}
    \caption{Lítio}
  \end{subfigure}
  \hfill
  \begin{subfigure}[t]{0.3\textwidth}
    \centering
    \frame{ \includegraphics[width=0.6\textwidth]{figuras/mineral_cobre.png}}
    \caption{Cobre}
  \end{subfigure}
  \caption{Exemplos de \textit{pixel art} de minerais gerados por IA
  (Gemini) e refinados manualmente.}
  \label{fig:minerais-ia}
\end{figure}

Para obter estes resultados, foi necessário especificar
características detalhadas no \textit{prompt} (instrução dada à IA),
como o estilo visual, a perspetiva isométrica e a paleta de cores. Um
exemplo do \textit{prompt} utilizado para gerar o ícone de ``Terras Raras'' foi:

\begin{quote}
  ``Terras Raras Pixel art icon of rare earth minerals, 32x32 pixels,
  multicolored iridescent crystalline rock cluster with rainbow
  shimmer effect (mix of green, blue, purple, pink hues), dark
  outlines, simple geometric facets showing 3D depth, clean retro RPG
  game item style matching these references, transparent background,
  isometric 3/4 view, magical glowing appearance''
\end{quote}

Outra decisão importante foi manter o fluxo de interação simples e repetível.
O utilizador pode regressar ao mapa após o questionário para rever objetos já
descobertos ou encontrar novos, permitindo aprendizagem progressiva sem impor
uma sequência rígida. Por fim, ao estruturar os dados de conteúdo (objetos,
minerais, origem e utilização) de forma consistente, o projeto facilita a
manutenção e a expansão para novos cenários, como a introdução de um nível
centrado numa escola.

\subsection{Conceito, regras e âmbito do jogo}

\subsubsection{Conceito}

O conceito central do ``Casa dos Minerais'' foca-se na exploração
interativa e autoguiada de um ambiente virtual, começando numa casa,
e podendo ser expandido para outros cenários, como uma escola.
Inspirado num formato \textit{point-and-click}, semelhante ao ``Club
Penguin'', o jogador tem liberdade para navegar pelo mapa e interagir
com objetos do quotidiano.

Ao clicar em objetos comuns (por exemplo, um frigorífico, uma televisão
ou uma torradeira), o jogador acede a informação sobre os minerais
associados ao seu fabrico. O objetivo é tornar visível a relação entre
recursos naturais e produtos do dia a dia, através de exemplos concretos
e familiares.

\subsubsection{Loop de gameplay}

O \textit{Casa dos Minerais} segue um ciclo simples e repetível,
pensado para alternar descoberta e reforço do conteúdo.
O jogador inicia num cenário (por exemplo, a casa) e explora o mapa,
interagindo com objetos do quotidiano para desbloquear informação
sobre os minerais/minérios associados.

Quando decide consolidar o que aprendeu, o jogador sai do cenário
e inicia o questionário ao clicar numa área específica (marcada pela cor azul).
O quiz é composto por 6 perguntas de escolha múltipla, com 4 opções
por pergunta e apenas 1 resposta correta.

\begin{figure}[H]
  \centering
  \includegraphics[width=0.55\textwidth]{figuras/quiz_area.png}
  \caption{Zona de transição onde o questionário é iniciado.
  Ao sair do mapa e clicar no ``chão azul'', começa o quiz.}
  \label{fig:chao-azul-quiz}
\end{figure}

Para avançar para o cenário seguinte, o jogador deve obter pelo menos
50\% de respostas corretas no questionário. Caso não atinja esse valor,
pode regressar ao mapa e tentar novamente após rever os conteúdos.

Após concluir o quiz, o jogador regressa ao mapa, podendo continuar a
exploração no mesmo cenário ou prosseguir para o próximo, caso cumpra
o critério de aprovação. Este ciclo promove aprendizagem progressiva,
mantendo sessões curtas e um fluxo consistente entre cenários.

\subsubsection{Regras e progressão}

As regras são simples. A exploração não impõe uma sequência fixa de objetos
nem penaliza o jogador por escolhas de percurso , cada um descobre à
sua maneira.

A progressão entre cenários depende do desempenho no questionário. Quando
o jogador não atinge o mínimo, pode voltar ao mapa, rever os conteúdos e
tentar novamente.

\subsubsection{Âmbito do jogo}

O âmbito do \textit{Casa dos Minerais} centra-se na descoberta de informação
sobre minerais/minérios em objetos do quotidiano e no reforço desse conteúdo
através de um mecanismo simples de questionário. O foco do projeto é oferecer
uma experiência curta, clara e educativa, adequada para uso autónomo ou em
contexto de sala de aula.

Não fazem parte do âmbito atual funcionalidades avançadas como inventário,
sistemas complexos de personalização, narrativa extensa, economia de jogo,
ou modos competitivos. A prioridade é a qualidade do conteúdo, a coerência
visual e a estabilidade da experiência em diferentes dispositivos.

\subsection{Como integrar outros públicos?}
\label{sec:other}

\subsubsection{Acessibilidade}

A acessibilidade deve fazer parte do design e não ser apenas um extra. Em
contexto web, as recomendações do W3C através da \textit{Web Accessibility
Initiative (WAI)} são a referência que seguimos.

Como o jogo é renderizado num \textit{canvas}, é importante garantir que a
informação relevante (por exemplo, textos descritivos e botões/ações) não
depende exclusivamente de pixels no ecrã. Sempre que possível, essa informação
deve existir também em elementos HTML semânticos, permitindo leitura por
tecnologias de apoio, como leitores de ecrã.

Do ponto de vista de interface, boas práticas incluem: contraste adequado entre
texto e fundo, tamanhos de letra legíveis, linguagem clara e consistente, e
feedback visual percetível quando ocorre uma interação. Atualmente, a interação
do jogo é feita apenas com o rato, pelo que a expansão para suporte de teclado e
toque (\textit{touch}) constitui uma melhoria relevante para aumentar a
acessibilidade em diferentes dispositivos.

\subsubsection{Internacionalização}

Para que o jogo funcione em diferentes línguas sem alterar a lógica do
programa, o texto e o conteúdo localizável são mantidos separados do
código. A estratégia adotada passa por separar texto e conteúdo localizável
do código, mantendo strings de interface e textos informativos (nomes de
objetos, minerais e descrições) em ficheiros externos, selecionados consoante
o idioma.

Esta abordagem facilita suportar, no mínimo, português e inglês, e permite
adicionar novos idiomas no futuro com impacto reduzido na implementação.
Embora os browsers disponibilizem tradução automática, a localização nativa
com recursos próprios oferece maior controlo terminológico e maior consistência
do conteúdo educativo.

\subsubsection{Informação semântica (HTML vs canvas)}

Quando o jogo é executado no navegador, grande parte do conteúdo visual é
renderizado num elemento HTML \texttt{canvas}. No entanto, o \texttt{canvas}
representa a informação como pixels, o que significa que tecnologias de apoio
(como leitores de ecrã) não conseguem interpretar diretamente o texto ou a
estrutura apresentada no ecrã.

Por este motivo, sempre que existe informação relevante para a aprendizagem
(por exemplo, descrições de minerais, botões de interface ou mensagens de
feedback), é preferível disponibilizá-la também através de HTML semântico,
com elementos apropriados. Isto melhora a acessibilidade e permite maior
compatibilidade com ferramentas de tradução e de apoio ao utilizador.

No contexto deste projeto, o conteúdo educativo não deve depender apenas da
representação gráfica. Ao separar a camada visual (canvas) da camada
informativa (HTML semântico), é possível manter a experiência de jogo e, ao
mesmo tempo, tornar o conteúdo acessível e reutilizável.

\begin{figure}[H]
  \centering
  \begin{tikzpicture}[
      node distance=1.2cm,
      every node/.style={align=center},
      box/.style={draw, rounded corners, minimum width=4.6cm, minimum
      height=1.1cm}
    ]

    \node[box] (canvas) {Canvas\\(sprites, mapa, animações)};
    \node[box, right=3.2cm of canvas] (html) {HTML semântico\\(texto,
    botões, descrições)};

    \node[box, below=1.6cm of $(canvas)!0.5!(html)$, minimum width=9.8cm] (ui)
    {Interface final no navegador\\(camada visual + camada informativa)};

    \draw[->] (canvas) -- (ui);
    \draw[->] (html) -- (ui);

  \end{tikzpicture}
  \caption{Separação entre a camada visual (canvas) e a camada informativa
    (HTML semântico), permitindo acessibilidade e melhor suporte a tecnologias
  de apoio.}
  \label{fig:canvas-html-semantico}
\end{figure}

\subsubsection{Independência de plataforma e performance}

O jogo precisa de correr em diferentes plataformas e em dispositivos com
capacidades variadas. Em contexto educativo, é comum haver restrições de
hardware e de rede, pelo que uma solução multiplataforma reduz barreiras de
acesso.

Atualmente, o jogo é disponibilizado no navegador, permitindo acesso imediato
sem instalação. No entanto, a base de código foi estruturada de forma a manter
a lógica do jogo separada de detalhes específicos da plataforma, tornando mais
simples uma futura adaptação para execução nativa noutros sistemas. Esta opção
preserva o investimento de desenvolvimento e facilita a expansão do projeto.

Do ponto de vista de performance, a execução em browser exige cuidados
particulares, como reduzir o peso de assets, evitar carregamentos desnecessários
e manter um \textit{loop} de atualização/renderização eficiente.
Em geral, a performance depende sobretudo de:
\begin{itemize}
  \item gestão de recursos gráficos (tamanho e número de sprites/tiles);
  \item otimização do ciclo de jogo (\textit{update}/\textit{draw});
  \item minimizar operações custosas durante a interação.
\end{itemize}
Estas medidas ajudam a garantir uma experiência fluida e consistente,
mesmo em dispositivos mais modestos.

\subsection{Planeamento e organização do trabalho}
Numa fase inicial, o planeamento do projeto foi realizado em conjunto,
com foco na definição do mapa, do fluxo do jogo e das principais
funcionalidades. A partir desse planeamento, as tarefas foram divididas
por áreas, de forma a permitir desenvolvimento em paralelo e reduzir
dependências.

A distribuição principal de responsabilidades foi a seguinte:
\begin{itemize}
  \item Torres: lógica principal do jogo e integração dos sistemas.
  \item Danilo: construção do mapa e elementos associados ao cenário
    (assets, organização do nível e interação com o ambiente).
  \item Guilherme: sistema de navegação e \textit{pathfinding}, bem como
    suporte ao design e implementação do questionário.
\end{itemize}

Com esta divisão, cada membro tinha um foco definido, e a coordenação
fazia-se pelo planeamento conjunto e pela integração progressiva do trabalho.

\section{Tech Stack}

Esta secção descreve as tecnologias usadas no desenvolvimento do
\textit{Casa dos Minerais} e as razões por trás de cada escolha.

\subsection{Linguagem (Odin)}

O \textit{Casa dos Minerais} usa Odin, uma linguagem compilada
com foco em simplicidade e controlo direto sobre recursos. Permite implementar
a lógica do jogo sem dependências excessivas, com um modelo de programação
próximo de C.

A integração de Odin com bibliotecas C facilitou o uso da raylib para gráficos
e multimédia, e simplifica a adaptação a diferentes plataformas sem reescrever
a lógica central.

\subsection{Biblioteca gráfica/multimédia (raylib)}

Para a execução nativa foi utilizada a raylib, integrada através do módulo
\texttt{vendor:raylib}. Esta biblioteca fornece a base do ciclo de aplicação
em desktop, incluindo criação de janela, controlo de \textit{frame rate},
câmara 2D, leitura de input do rato e desenho dos elementos visuais.

No código nativo (\texttt{native/main.odin}), a raylib é usada para inicializar
e executar o \textit{loop} principal e para dar suporte à navegação no mapa.
Exemplos de chamadas relevantes incluem \texttt{InitWindow} e
\texttt{SetTargetFPS} (configuração), \texttt{BeginMode2D} e
\texttt{GetScreenToWorld2D} (câmara e conversão de coordenadas), e funções de
desenho como \texttt{DrawRectangle} e \texttt{DrawCircleV}. Esta abordagem
permitiu validar rapidamente a lógica do jogo e o comportamento de interação
num ambiente local simples de executar.

No estado atual do projeto, a raylib é usada sobretudo como base gráfica e de
input no alvo nativo. Na versão web, a camada de áudio é gerida pelo
JavaScript através da Web Audio API,
o que permite adaptar a gestão de som às restrições e APIs do navegador.

\subsection{Build web / WebAssembly (WASM)}

Para disponibilizar o jogo no navegador, a lógica principal é compilada em Odin
para o alvo \texttt{js\_wasm32}, gerando o módulo \texttt{index.wasm}. Este
processo é automatizado através de \texttt{build.sh} e \texttt{build.bat}, que
também copiam o \texttt{odin.js} (runtime de Odin) para a pasta \texttt{web/}.

De forma resumida, o fluxo de build web é:
\begin{enumerate}
  \item copiar \texttt{core/sys/wasm/js/odin.js} para \texttt{web/odin.js};
  \item compilar o projeto com \texttt{odin build . -target:js\_wasm32};
  \item gerar a saída em \texttt{web/index.wasm}.
\end{enumerate}

No arranque da aplicação web, o \texttt{odin.js} é carregado em
\texttt{web/index.html} e o WASM é inicializado em \texttt{web/index.js}
através de \texttt{odin.runWasm("index.wasm", ...)}. A partir desse ponto,
o JavaScript gere a camada de interface e renderização em \textit{canvas},
enquanto a lógica do jogo permanece no módulo compilado em WASM.

Como o navegador impõe restrições de acesso a ficheiros locais para módulos
WASM, a execução é feita através de um servidor HTTP estático na pasta
\texttt{web/}, conforme descrito no \texttt{README.md}.

\subsection{Scripts/ferramentas de build e colaboração (Git, etc.)}

O desenvolvimento foi suportado por um conjunto de ferramentas e scripts para
facilitar colaboração e tornar o processo de build reprodutível. Para controlo
de versões foi utilizado Git, com alojamento no GitHub, permitindo trabalho em
paralelo e histórico completo de alterações através de \textit{commits}. Esta
abordagem simplificou a integração do trabalho do grupo e reduziu conflitos
durante a implementação.

Para automatizar tarefas repetitivas, o repositório inclui scripts de build,
nomeadamente \texttt{build.sh} e \texttt{build.bat}. Estes scripts tratam da
compilação para WebAssembly e da preparação da pasta \texttt{web/}, incluindo
a cópia do runtime \texttt{odin.js} e a geração do ficheiro \texttt{index.wasm}.
Desta forma, o processo de compilação torna-se consistente entre sistemas
operativos e reduz erros associados a passos manuais.

Para testar localmente, a versão web é servida por um servidor
HTTP estático (conforme descrito no \texttt{README.md}), contornando as
restrições do navegador relativas ao carregamento de módulos WASM a partir do
sistema de ficheiros.

\begin{figure}[H]
  \centering
  \includegraphics[width=\textwidth]{figuras/fluxo_build_execucao.png}
  \caption{Fluxo simplificado de build e execução da versão web do jogo,
    desde o código-fonte e scripts de compilação até ao carregamento do WASM
  no navegador.}
  \label{fig:fluxo-build-execucao}
\end{figure}

%%%

% \begin{figure}[!htb]
%   \centering
%   \caption{Exemplo de Figura flutuante (\textit{in}
%   \url{http://en.wikipedia.org/wiki/Class_diagram})}
%   \includegraphics[width=13cm]{exemploFig}
%   \label{fig:exemplofig}
% \end{figure}
%
% \begin{center}
%
%   \captionof{figure}{Exemplo de Figura não flutuante (\textit{in}
%   \url{http://en.wikipedia.org/wiki/Class_diagram})}
%   \includegraphics[width=\textwidth]{exemploFig}
%   %\includegraphics[width=0.5\textwidth]{exemploFig}
%   \label{fig:exemplofigFixa}
% \end{center}

% \lstinputlisting[language=java,abovecaptionskip=20pt,caption={Exemplo
%     de listagem de parte de um ficheiro Java na pasta
% listagens.},{label=lst:javaparcial},firstline=38,lastline=44]{listagens/quicksort.java}

% \noindent
% \begin{minipage}{\linewidth}
%   \captionof{table}{Uma tabela de exemplo não flutuante} % título da tabela
%   \begin{tabular}{l l c r} % duas colunas à esquerda (l l), uma ao
%     % centro (c) e uma à direita (r) (4 colunas)
%     \hline\hline %insere duas linhas horizontais
%     Caso & Método 1 & Método 2 & Método 3 \\ [0.5ex] % insere tabela
%     \hline % insere uma linha horizontal
%     1 & 50 & 837 & 970 \\ % insure corpo da tabela
%     2 & 47 & 877 & 230 \\
%     3 & 31 & 25 & 415 \\
%     4 & 35 & 144 & 2356 \\
%     5 & 45 & 300 & 556 \\ [1ex] % [1ex] adiciona espaço vertical
%     \hline %insere uma linha horizontal
%   \end{tabular}
%   \label{tab:casoFIXO} % é utilizada para referir a tabela no texto
%   \\ %% linha em braco
%   \textit{Nota}: uma nota. remover se não for necessária.\\
% \end{minipage}
%
% %baseado num exemplo em http://www1.maths.leeds.ac.uk/latex/TableHelp1.pdf
% \begin{table}[!htb]
%   \caption{Uma tabela de exemplo}
%   \begin{tabular}{l l c r} % duas colunas à esquerda (l l), uma ao
%     % centro (c) e uma à direita (r) (4 colunas)
%     \hline\hline %insere duas linhas horizontais
%     Caso & Método 1 & Método 2 & Método 3 \\ [0.5ex] % insere tabela
%     \hline % insere uma linha horizontal
%     1 & 50 & 837 & 970 \\ % insere corpo da tabela
%     2 & 47 & 877 & 230 \\
%     3 & 31 & 25 & 415 \\
%     4 & 35 & 144 & 2356 \\
%     5 & 45 & 300 & 556 \\ [1ex] % [1ex] adiciona espaço vertical
%     \hline %insere uma linha horizontal
%   \end{tabular}
%   \label{tab:caso} % é utilizada para referir a tabela no texto
%   \\ %% linha em braco
%   \textit{Nota}: uma nota. remover se não for necessária.\\
% \end{table}
