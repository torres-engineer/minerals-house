\chapter{Desenvolvimento}
\label{ch:develop}

\section{Enquadramento teórico}

\subsection{``Estado da arte''}

«Inspirado por jogos educativos como \textcite{briefcasegame},
[...] o objetivo é aumentar a literacia sobre recursos
geológicos, demonstrando de forma lúdica e visual a ligação direta
entre a geologia e a nossa vida diária.»

% O nosso toque pessoal, point click tipo Club Penguin...

\subsection{Os minerais}
\label{sec:minerals}

Eles andam por aí espalhados, mas tu nem pensas neles: os minerais.
Matéria-prima e recurso importante para diversos setores como a
indústria química, da construção civil, da agricultura e da
energia. Mas é que estão presentes em todo o lado mesmo:

\begin{itemize}
  \item Se estás a ler isto através de um aparelho eletrónico,
    esquece, desde o cobre e ouro dos cabos elétricos até ao
    silício dos microchips;
  \item mas mesmo lendo numa folha de papel, é capaz dessa folha
    ter vestígios de cádmio, cobalto e titânio.
\end{itemize}

As frutas e vegetais que consomes, além das vitaminas, também têm os
minerais presentes no solo. Minerais são indispensáveis para a
nossa vida e para a sociedade.

O objetivo principal deste jogo é ensinar onde é que estes
minerais, que existem naturalmente no nosso planeta, estão
presentes nas coisas que usamos todos os dias. Além isso,
ensinamos qual a utilidade de certo mineral numa ocasião específica.
Mas basta isso?

Uma das razões para a qual estes minerais ficam escondidos dos
consumidores é porque, tu vais à loja e compras o telemóvel, mas o
processo de produção desse telemóvel, da extração dos minerais da
terra, da montagem do equipamento, até ao produto final, é
totalmente ofuscado. Existe uma alienação entre as nossas vidas e a
geologia — os minerais — e todo o trabalho humano que leva à
realização dos teus bens. Simplesmente o telemóvel não apareceu na
estante da loja por pozinhos de perlimpimpim. O telemóvel é visto
como uma coisa e não como fruto de mineração e do trabalho. Tudo
isto já havia sido dito na \ref{ch:intro}.

Usamos o jogo para conectar também o mineral, ao país de origem
(geografia), e ao trabalho e/ou processo necessário para esse
material estar disponível.

\subsection{Gamificação na pedagogia}

A ``Casa dos Minerais'' utilizará elementos de gamificação para
estruturar a aprendizagem, como a progressão por diferentes níveis
e a inclusão de minijogos, por exemplo, um questionário no final de
cada nível.

No entanto, a abordagem pedagógica deste projeto foca-se em usar a
brincadeira como uma forma de descoberta e reflexão. O objetivo
principal não é criar um sistema de avaliação rigoroso; o foco é a
alegria de aprender. A aprendizagem é mais importante do que uma
pontuação final ou a rapidez com que se termina o jogo.

Desta forma, elementos como a pontuação ou um temporizador,
servirão como um incentivo opcional (conforme detalhado na secção
\ref{sec:future}) e não como o foco central da experiência.

\subsection{Para quem é o jogo?}
\label{sec:forwho}

O jogo têm como alvo principal as crianças e jovens, e serviria de
apoio para o que eles estão a aprender nas aulas de Estudo de Meio
e Ciências Naturais (no contexto de Portugal). Mas a verdade é que
todos merecem aprender, por isso precisamos de pensar em fazer um
jogo para todas as faixas etárias. Desde que tenha o interesse em
adquirir conhecimento acerca dos minerais.

Mas esta inclusão não pode parar na idade. Existem pessoas com
limitações a nível físico (controlo, mobilidade), cognitivo
(pensamento, memória, processamento de informação), sensorial,
visão, audição e/ou fala, e elas também merecem de ter o direito à
sabedoria (e à diversão de jogar).

O que pode parecer trabalhoso ter de se preocupar com todos na
realidade pode ser algo bem simples de alcançar, por exemplo,
escolhendo a tipografia correta, baseado no contraste da cor de
letra, do tamanho de letra, e no tipo de letra. Especialmente no
ambiente de navegadores web, estes estão muito preparados para
serem úteis e utilizáveis por todos, basta aos programadores
fazerem o seu trabalho corretamente.

\subsection{Aprendizagem por descoberta e contextualização}
%Como “ancorar” conceitos abstratos (minerais) em exemplos concretos (objetos em casa/escola). Dá base teórica para o vosso design de cenário
A aprendizagem por descoberta baseia-se na ideia de que o aluno compreende 
melhor quando participa ativamente no processo, explorando, 
pensando por ele mesmo e construindo ligações por iniciativa própria. 
Em vez de receber a informação de forma expositiva, 
o utilizador é colocado num ambiente interativo onde a curiosidade 
e a exploração conduzem naturalmente à aprendizagem.

No caso do Casa dos Minerais, 
esta abordagem é reforçada pela contextualização: conceitos abstratos como “mineral”, 
“recurso” ou “matéria-prima” são ancorados em exemplos 
concretos e familiares, como objetos existentes numa casa ou numa escola. 
Ao interagir com esses objetos, o jogador passa a associar materiais 
e tecnologias do quotidiano à sua origem geológica, 
tornando o conteúdo mais acessível e memorável. 
Esta ligação direta entre o que o utilizador já reconhece 
(um frigorífico, uma torradeira, um computador) e o 
conceito que se pretende ensinar (minerais e o seu papel no fabrico) 
reduz a distância entre teoria e prática.

\subsection{Ciclo pedagógico do jogo}
%Justifica o vosso fluxo: explorar, ler info, depois quiz de 6 perguntas (reforço/avaliação formativa).

O ciclo do jogo foi feito de maneira a combinar aprendizagem por descoberta 
com reforço através de avaliação formativa. 
O jogo organiza-se em duas fases complementares: uma fase de exploração guiada pela 
curiosidade e uma fase de consolidação através de um questionário curto. 
Esta estrutura permite que o utilizador primeiro compreenda os conceitos no seu 
contexto e, de seguida, os recupere ativamente, promovendo retenção.

Na fase inicial, o utilizador explora o cenário e interage com objetos do quotidiano. 
Cada interação mostra informação sobre os minerais associados, 
permitindo aprender de forma simples e ao próprio ritmo, sem uma ordem obrigatória.

Quando termina a exploração, o utilizador sai do cenário 
e pode iniciar o quiz. O questionário tem seis perguntas 
e serve para verificar e reforçar o que foi aprendido, 
através de recordação direta dos conteúdos.

No final do quiz, o jogo reinicia e o ciclo pode ser repetido. 
Isto incentiva a revisão dos conteúdos 
e ajuda a consolidar as ligações entre objetos 
e minerais, sendo adequado para sessões curtas em contexto educativo.

\subsection{Questionários como reforço}
%Porque quizzes curtos funcionam (recall ativo, reforço, feedback imediato). Perfeito para explicar porquê 6 perguntas e repetição do jogo.

Os questionários são uma forma eficaz de reforçar a aprendizagem porque 
obrigam o utilizador a recuperar a informação da memória, em vez de apenas 
a reconhecer. Este processo, muitas vezes designado por recordação ativa, 
tende a melhorar a retenção a médio e longo prazo, especialmente quando é 
feito pouco tempo depois do contacto inicial com o conteúdo.

No Casa dos Minerais, o quiz surge após a fase de exploração para consolidar 
as associações entre objetos do quotidiano e os minerais envolvidos no seu fabrico. 
Como as perguntas aparecem depois do utilizador ter visto a informação no jogo, 
o questionário funciona como avaliação formativa: identifica o que foi compreendido, 
reforça os conteúdos principais sem transformar o jogo num teste “pesado”.

A escolha de um questionário curto com seis perguntas procura equilibrar 
duas necessidades. Por um lado, é suficientemente pequeno para manter o 
ritmo do jogo e evitar fadiga, tornando-o adequado a sessões curtas. Por 
outro, é suficiente para obrigar a várias recuperações de informação, 
reforçando o essencial do conteúdo explorado. No final, o reinício do ciclo 
incentiva a repetição do processo, permitindo ao utilizador voltar a explorar, 
rever informação e melhorar no questionário, o que contribui para aprendizagem progressiva
\subsection{Qualidade do conteúdo e fontes}
%Contraste, tamanho de texto, linguagem simples, input por rato/teclado. Mesmo que implementem pouco, dá para discutir “boas práticas”.

A qualidade do conteúdo é fundamental para garantir que o jogo seja acessível,
envolvente e educativo. Para conseguir isso, é importante seguir boas práticas
e apresentação de informação, como:

\begin{itemize}
  \item Fontes fiáveis: as associações entre objetos do quotidiano e minerais
  devem ser baseadas em fontes reconhecidas (por exemplo, recursos educativos oficiais,
  instituições científicas ou bases de dados geológicas), evitando informação não verificada.
  Neste trabalho foram privilegiadas fontes institucionais e educativas como a USGS
  \textcite{usgs_education_mrp,usgs_minerals101} e a Minerals Education Coalition
  \textcite{mec_minerals_in_your_life}.

  \item Confirmação e coerência: sempre que possível, a informação é cruzada entre mais do que uma referência,
  reduzindo erros e ambiguidades.

  \item Consistência terminológica: utilização consistente dos termos (por exemplo, “mineral” vs “minério”),
  nomes dos minerais, e forma de escrita ao longo do jogo, para evitar confusão no utilizador.

  \item Estrutura uniforme do conteúdo: manter um formato semelhante nas descrições (ex.: mineral associado,
  uso no objeto e nota curta), facilitando a leitura e a comparação entre objetos.

  \item Linguagem adequada ao público-alvo: textos claros e diretos, com o mínimo de palavreado técnico,
  para garantir compreensão por crianças e jovens, sem perder rigor.
\end{itemize}


\section{Metodologia}

Para a implementação, o projeto baseia-se em tecnologias de
multiplataforma, tirando partido da linguagem de programação Odin.
Para a parte gráfica escolhemos a Raylib, integrada com Odin. Esta
combinação permite criar gráficos 2D interativos com desempenho
nativo e exportação para WebAssembly (WASM), garantindo execução
fluida diretamente no navegador.

\subsection{Identificação dos minerais usados no quotidiano}

Minerais são todas as substâncias naturais formadas por processos
geológicos que normalmente consegues encontrar na crosta da terra.

A telhado das casas podem ser feitos com argila, os vidros com areia.
Na cozinha, tens pratos que podem conter areia, calcário, feldspato e
argila. O sal vem do sal-gema. Uma panela pode ser de ferro, níquel e
crómio. O frigorifico pode ser alumínio e cobre, e derivados de
petróleo. Na cozinha também pode ser usado gás natural. Existem
fotografias que usam prata e bronze. Um vaso de flores pode ser de
argila. Um livro pode ter vestígios de cadmio, cobalto e titânio. Uma
lâmpada pode ter níquel, alumínio e chumbo. Um relógio pode ter
silício, crómio, ferro e níquel. Um espelho terá quartzo, alumínio e
prata. O papel higiénico gesso e calcário. A pasta de dentes tem fluorite.

O mais impressionante talvez são os bens tecnológicos como
televisões/monitores, computadores e telemóveis, tão presentes no
nosso dia"-a"-dia e que os levamos de um lado para o outro connosco.
Estes podem conter zinco, crómio, silício, alumínio, cobre, estanho,
manganês, cobalto, bário, platina, tungsténio, arsénio, lítio, ouro, prata, ...

\subsection{Critérios de seleção de objetos (casa/escola)}

A seleção de objetos para os níveis ``casa'' e ``escola'' foi guiada por critérios 
pedagógicos e práticos, de forma a garantir que o conteúdo 
fosse relevante, compreensível e adequado ao formato do jogo. 
Em primeiro lugar, foram priorizados objetos comuns e facilmente 
reconhecíveis, pois isso reduz a barreira de entrada e ajuda o 
utilizador a relacionar o conteúdo com a sua experiência do dia a dia. 
Em segundo lugar, deu-se preferência a objetos que permitissem associações 
claras com minerais/minérios utilizados no seu fabrico, 
garantindo que cada interação tivesse valor educativo.

Adicionalmente, procurou-se diversidade de categorias (por exemplo, cozinha, 
lavandaria e eletrónica), evitando repetição e permitindo abordar diferentes 
utilizações de recursos minerais. Do ponto de vista do design do jogo, 
os objetos selecionados também tiveram de ser representáveis visualmente 
e inseridos no cenário de forma coerente, com áreas de interação claras 
e sem sobrecarregar o mapa com demasiados pontos clicáveis. 

Por fim, a existência de dois contextos (casa e escola) permite variar o 
conteúdo mantendo o mesmo fluxo de interação. O nível da casa favorece 
objetos ligados a habitação e consumo doméstico, enquanto o nível da escola 
introduz elementos típicos do ambiente escolar. Esta progressão por contexto 
facilita aprendizagem comparativa e incentiva o utilizador a explorar 
novamente para descobrir novas associações entre objetos e minerais.

\subsection{Critérios de seleção de objetos (casa/escola)}

A seleção de objetos para os níveis ``casa'' e ``escola'' foi guiada por critérios 
pedagógicos e práticos, de forma a garantir que o conteúdo 
fosse relevante, compreensível e adequado ao formato do jogo. 
Em primeiro lugar, foram priorizados objetos comuns e facilmente 
reconhecíveis, pois isso reduz a barreira de entrada e ajuda o 
utilizador a relacionar o conteúdo com a sua experiência do dia a dia. 
Em segundo lugar, deu-se preferência a objetos que permitissem associações 
claras com minerais/minérios utilizados no seu fabrico, 
garantindo que cada interação tivesse valor educativo. 

Adicionalmente, procurou-se diversidade de categorias (por exemplo, cozinha, 
lavandaria, eletrónica e música), evitando repetição e permitindo abordar 
diferentes utilizações de recursos minerais. Do ponto de vista do design do jogo, 
os objetos selecionados também tiveram de ser representáveis visualmente 
e inseridos no cenário de forma coerente, com áreas de interação claras 
e sem sobrecarregar o mapa com demasiados pontos clicáveis. 

Para suportar esta seleção de forma consistente, os objetos e minerais foram 
representados em dados estruturados, permitindo reutilização e extensão do 
conteúdo sem alterar a lógica do jogo. Cada objeto inclui um nome, uma 
categoria e uma lista de minerais/minérios associados, onde cada mineral 
regista a sua origem e a função no objeto. Esta estrutura facilita a 
manutenção do conteúdo e permite escalar o projeto para novos cenários 
(como o nível de escola), mantendo o mesmo formato. 

\begin{lstlisting}[language=json, caption={Excerto do dataset de objetos e minerais.}, label={lst:dataset-objetos}]
{
    {
      "name": "Fogão Elétrico",
      "category": "Cozinha",
      "minerals": [
        {
          "name": "Ferro",
          "origin": "Austrália, Brasil, China",
          "use": "Corpo do fogão e resistências"
        },
        {
          "name": "Crómio",
          "origin": "África do Sul, Cazaquistão",
          "use": "Resistências"
        },
        {
          "name": "Níquel",
          "origin": "Indonésia, Filipinas, Rússia",
          "use": "Revestimento"
        }
      ]
    }
}
\end{lstlisting}

\subsection{Comparação com outros jogos}

Em comparação com o \textcite{briefcasegame}, o \textit{Casa dos Minerais} 
aposta numa abordagem mais exploratória. O jogador percorre um cenário 
(point-and-click) e encontra objetos distribuídos no ambiente, 
descobrindo informação associada a cada interação. No \textit{BriefCase game}, 
a dinâmica é mais direta e centrada na associação imediata entre objeto 
e mineral, com menor ênfase na exploração espacial e no contexto do objeto. 
Assim, o nosso jogo privilegia a contextualização (objeto no seu ``habitat'') 
antes do momento de verificação através do questionário.

Relativamente ao \textcite{min4kids-minerals-house}, existe uma proximidade 
no objetivo educativo (relacionar minerais com objetos do quotidiano), 
mas a apresentação difere. Esse recurso tende a ser mais informacional, 
funcionando como consulta/explicação, enquanto o \textit{Casa dos Minerais} 
procura integrar a informação num fluxo de jogo, combinando descoberta no mapa 
com um quiz de escolha múltipla (4 opções, 1 correta) para reforço do conteúdo. 
Desta forma, o utilizador alterna entre explorar e testar o que aprendeu.

No caso do \textcite{wordwall-rochas-minerais}, o foco está num conjunto de 
atividades curtas e simples, normalmente com interações rápidas e regras 
minimalistas. Embora sejam úteis como exercícios imediatos, tendem a ter menos 
camada de contexto e menos continuidade. Em contraste, o nosso projeto mantém 
um cenário persistente, permitindo aprender por exploração e regressar ao mapa 
após o questionário para rever ou descobrir novos objetos.

Em síntese, o \textit{Casa dos Minerais} distingue-se por juntar:
\begin{enumerate}
  \item exploração contextualizada em cenários familiares; e
  \item validação por questionário curto, suportando a repetição do ciclo para
  consolidação progressiva da aprendizagem.
\end{enumerate}



\begin{table}[H]
\centering
\caption{Comparação entre recursos/jogos educativos sobre minerais.}
\label{tab:comparacao-jogos}
\begin{tabular}{p{3.2cm} p{3.6cm} p{4.0cm} p{4.0cm}}
\hline
\textbf{Recurso} & \textbf{Tipo de interação} & \textbf{Contexto (cenário)} & \textbf{Avaliação / feedback} \\
\hline
\textcite{briefcasegame} 
& Associação direta objeto--mineral 
& Pouco contexto espacial; foco na correspondência 
& Feedback imediato da associação \\
\hline
\textcite{min4kids-minerals-house} 
& Consulta/informação com interação limitada 
& Conteúdo mais informacional; menor componente de ``jogo'' 
& Sem validação estruturada; reforço principalmente pela leitura \\
\hline
\textcite{wordwall-rochas-minerais} 
& Mini-jogos simples (atividades curtas) 
& Contexto reduzido; tarefas isoladas 
& Feedback rápido por exercício, geralmente sem progressão contínua \\
\hline
\textit{Casa dos Minerais} 
& Exploração point-and-click + quiz de escolha múltipla 
& Cenários familiares (casa/escola) com objetos distribuídos 
& Quiz curto (6 perguntas, 4 opções, 1 correta) e possibilidade de voltar a explorar \\
\hline
\end{tabular}
\end{table}


\subsection{Diferenças de abordagem (exploração vs associação direta)}

Uma diferença central entre jogos educativos sobre minerais está na forma como 
o conhecimento é apresentado e praticado. Numa abordagem de \textbf{exploração}, 
o utilizador aprende ao interagir com um ambiente e ao descobrir conteúdo no 
contexto em que os objetos aparecem. Isto favorece a contextualização, pois o 
objeto não surge isolado: está inserido num cenário familiar (por exemplo, uma 
casa ou uma escola), o que facilita a ligação entre o conceito abstrato 
(mineral/minério) e o exemplo concreto (objeto do quotidiano).

Em contraste, numa abordagem de \textbf{associação direta}, o objetivo é mais 
imediato: o utilizador recebe um objeto (ou um nome/imagem) e deve associá-lo 
ao mineral correto. Este formato tende a ser mais rápido e focado em treino, 
mas pode reduzir a componente de descoberta e o contexto de uso do objeto, 
dependendo da implementação.

No \textit{Casa dos Minerais}, a exploração funciona como fase inicial de 
aprendizagem, permitindo observar e compreender as associações objeto--mineral 
antes da validação. Em seguida, o questionário de escolha múltipla (4 opções, 
1 correta) força a recordação ativa dessas associações, combinando os benefícios 
de ambos os estilos: primeiro contextualiza-se, depois pratica-se e verifica-se.

\subsection{Limitações observadas e decisões adotadas no projeto}

Durante a análise de recursos e ao longo do desenvolvimento, 
foram identificadas limitações comuns em abordagens educativas sobre minerais. 
Uma delas é a tendência para conteúdos demasiado expositivos, onde o utilizador 
lê informação mas tem pouca oportunidade de a aplicar. Outra limitação é a 
associação direta sem contexto, que pode treinar a memorização de pares 
objeto--mineral, mas nem sempre ajuda a compreender a relevância desses minerais 
no quotidiano.

Para responder a estas limitações, o \textit{Casa dos Minerais} adotou decisões 
de design que equilibram simplicidade com contexto. Em vez de depender apenas 
de exercícios isolados, o jogo organiza a aprendizagem em dois momentos: 
(1) exploração num cenário familiar, onde a informação surge ligada a objetos 
do dia a dia; e (2) um questionário curto de escolha múltipla para reforço e 
validação, mantendo o ritmo e evitando fadiga.

Do ponto de vista visual, também existiram limitações na produção de assets. 
Numa fase inicial, foi testada a geração de elementos gráficos com o Gemini, 
mas os resultados não cumpriam requisitos técnicos essenciais para um mapa 
baseado em tiles, nomeadamente a disponibilização de um \textit{tileset} com 
grelha 32$\times$32\,px. Apesar de visualmente apelativas, as imagens geradas 
surgiam tipicamente em dimensões muito grandes e sem organização reutilizável, 
o que dificultava recorte, alinhamento e integração consistente no cenário. 
Por esse motivo, optou-se por adquirir um conjunto de assets estruturados, 
garantindo compatibilidade com o formato do mapa, consistência visual e uma 
experiência mais polida.

Outra decisão importante foi manter o fluxo de interação simples e repetível. 
O utilizador pode regressar ao mapa após o questionário para rever objetos já 
descobertos ou encontrar novos, permitindo aprendizagem progressiva sem impor 
uma sequência rígida. Por fim, ao estruturar os dados de conteúdo (objetos, 
minerais, origem e utilização) de forma consistente, o projeto facilita a 
manutenção e a expansão para novos cenários, como a introdução de um nível 
centrado numa escola.

\begin{figure}[H]
  \centering
  \begin{subfigure}[t]{0.58\textwidth}
    \centering
    \includegraphics[width=\textwidth]{figuras/gemini_assets.jpeg}
    \caption{Asset gerado automaticamente (Gemini): imagem visualmente apelativa, 
    mas sem estrutura de \textit{tileset} e sem grelha 32$\times$32\,px, 
    dificultando recorte, alinhamento e reutilização no mapa.}
    \label{fig:gemini-gerado}
  \end{subfigure}
  \hfill
  \begin{subfigure}[t]{0.38\textwidth}
    \centering
    \includegraphics[width=\textwidth]{figuras/paid_assets.jpeg}
    \caption{Tileset estruturado: elementos organizados e dimensionados, 
    facilitando integração, consistência visual e reutilização no cenário.}
    \label{fig:tileset-estruturado}
  \end{subfigure}

  \caption{Comparação entre geração automática e assets estruturados para jogos 2D.
  Apesar da qualidade estética, a ausência de formato e escala (ex.: 32$\times$32\,px) 
  na geração automática limitou a utilização prática no projeto.}
  \label{fig:comparacao-gemini-tileset}
\end{figure}

A Figura~\ref{fig:comparacao-gemini-tileset} ilustra que a principal limitação 
não foi estética, mas sim técnica: o projeto exigia assets reutilizáveis em 
tilesets com dimensões uniformes (32$\times$32\,px) para montagem eficiente do mapa.

\subsection{Conceito, regras e âmbito do jogo}

\subsubsection{Conceito}

O conceito central do ``Casa dos Minerais'' foca-se na exploração 
interativa e autoguiada de um ambiente virtual, começando numa casa, 
e podendo ser expandido para outros cenários, como uma escola. 
Inspirado num formato \textit{point-and-click}, semelhante ao ``Club 
Penguin'', o jogador tem liberdade para navegar pelo mapa e interagir 
com objetos do quotidiano.

Ao clicar em objetos comuns (por exemplo, um frigorífico, uma televisão 
ou uma torradeira), o jogador acede a informação sobre os minerais 
associados ao seu fabrico. O objetivo é tornar visível a relação entre 
recursos naturais e produtos do dia a dia, através de exemplos concretos 
e familiares.

\subsubsection{Loop de gameplay}

O \textit{Casa dos Minerais} segue um ciclo simples e repetível, 
pensado para alternar descoberta e reforço do conteúdo. 
O jogador inicia num cenário (por exemplo, a casa) e explora o mapa, 
interagindo com objetos do quotidiano para desbloquear informação 
sobre os minerais/minérios associados.

Quando decide consolidar o que aprendeu, o jogador sai do cenário 
e inicia o questionário ao clicar numa área específica (marcada pela cor azul). 
O quiz é composto por 6 perguntas de escolha múltipla, com 4 opções 
por pergunta e apenas 1 resposta correta.

\begin{figure}[H]
  \centering
  \includegraphics[width=0.55\textwidth]{figuras/quiz_area.png}
  \caption{Zona de transição onde o questionário é iniciado. 
  Ao sair do mapa e clicar no ``chão azul'', começa o quiz.}
  \label{fig:chao-azul-quiz}
\end{figure}

Para avançar para o cenário seguinte, o jogador deve obter pelo menos 
50\% de respostas corretas no questionário. Caso não atinja esse valor, 
pode regressar ao mapa e tentar novamente após rever os conteúdos.

Após concluir o quiz, o jogador regressa ao mapa, podendo continuar a 
exploração no mesmo cenário ou prosseguir para o próximo, caso cumpra 
o critério de aprovação. Este ciclo promove aprendizagem progressiva, 
mantendo sessões curtas e um fluxo consistente entre cenários.

\subsubsection{Regras e progressão}

As regras do jogo privilegiam clareza e acessibilidade. A exploração 
não impõe uma sequência fixa de objetos nem penaliza o jogador por 
escolhas de percurso, permitindo diferentes estratégias de descoberta.

A progressão entre cenários é condicionada pelo desempenho no momento 
de consolidação (questionário). Quando o jogador não atinge o critério 
mínimo, a experiência incentiva a revisão do conteúdo através do regresso 
ao mapa, promovendo repetição e melhoria incremental.

\subsubsection{Âmbito do jogo}

O âmbito do \textit{Casa dos Minerais} centra-se na descoberta de informação 
sobre minerais/minérios em objetos do quotidiano e no reforço desse conteúdo 
através de um mecanismo simples de questionário. O foco do projeto é oferecer 
uma experiência curta, clara e educativa, adequada para uso autónomo ou em 
contexto de sala de aula.

Não fazem parte do âmbito atual funcionalidades avançadas como inventário, 
sistemas complexos de personalização, narrativa extensa, economia de jogo, 
ou modos competitivos. A prioridade é a qualidade do conteúdo, a coerência 
visual e a estabilidade da experiência em diferentes dispositivos.


\subsection{Como integrar outros públicos?}

\subsubsection{Acessibilidade}

Para que o jogo seja útil a diferentes públicos, a acessibilidade deve ser 
considerada como parte do design e não apenas como um extra. Em contexto web, 
existem recomendações consolidadas pelo W3C através da \textit{Web Accessibility 
Initiative (WAI)}, que servem como referência para tornar interfaces mais 
inclusivas.

Como o jogo é renderizado num \textit{canvas}, é importante garantir que a 
informação relevante (por exemplo, textos descritivos e botões/ações) não 
depende exclusivamente de pixels no ecrã. Sempre que possível, essa informação 
deve existir também em elementos HTML semânticos, permitindo leitura por 
tecnologias de apoio, como leitores de ecrã.

Do ponto de vista de interface, boas práticas incluem: contraste adequado entre 
texto e fundo, tamanhos de letra legíveis, linguagem clara e consistente, e 
feedback visual percetível quando ocorre uma interação. Atualmente, a interação 
do jogo é feita apenas com o rato, pelo que a expansão para suporte de teclado e 
toque (\textit{touch}) constitui uma melhoria relevante para aumentar a 
acessibilidade em diferentes dispositivos.

  \subsubsection{Internacionalização}
A internacionalização é importante para que o jogo possa ser utilizado por 
públicos com diferentes línguas e contextos culturais, sem alterar a lógica do 
programa. A ideia central é separar texto e conteúdo localizável do código e da 
interface, permitindo traduzir o jogo através de ficheiros de recursos e de uma 
estrutura consistente de chaves.

No \textit{Casa dos Minerais}, uma estratégia adequada passa por manter as 
strings de interface e os textos informativos (nomes de objetos, minerais e 
descrições) em ficheiros externos, selecionados consoante o idioma. Esta 
abordagem facilita suportar, no mínimo, português e inglês, e torna possível 
adicionar outros idiomas no futuro com impacto reduzido na implementação.

Em contexto web, os browsers oferecem funcionalidades de tradução automática, 
mas depender exclusivamente dessas ferramentas pode causar perda de precisão, 
especialmente em termos científicos. Por isso, uma solução de localização 
nativa (com strings próprias) permite maior controlo sobre terminologia e 
consistência do conteúdo, preservando o rigor educativo do jogo.

\subsubsection{Informação semântica (HTML vs canvas)}

Quando o jogo é executado no navegador, grande parte do conteúdo visual é 
renderizado num elemento HTML \texttt{canvas}. No entanto, o \texttt{canvas} 
representa a informação como pixels, o que significa que tecnologias de apoio 
(como leitores de ecrã) não conseguem interpretar diretamente o texto ou a 
estrutura apresentada no ecrã.

Por este motivo, sempre que existe informação relevante para a aprendizagem 
(por exemplo, descrições de minerais, botões de interface ou mensagens de 
feedback), é preferível disponibilizá-la também através de HTML semântico, 
com elementos apropriados. Esta abordagem 
melhora a acessibilidade, facilita indexação e permite maior compatibilidade 
com ferramentas de tradução e de apoio ao utilizador.

No contexto deste projeto, esta distinção é particularmente importante porque 
o conteúdo educativo não deve depender apenas da representação gráfica. Ao 
separar a camada visual (canvas) da camada informativa (HTML semântico), 
torna-se possível manter a experiência de jogo e, ao mesmo tempo, aumentar a 
inclusão e a reutilização do conteúdo.

\begin{figure}[H]
  \centering
  \begin{tikzpicture}[
    node distance=1.2cm,
    every node/.style={align=center},
    box/.style={draw, rounded corners, minimum width=4.6cm, minimum height=1.1cm}
  ]

    \node[box] (canvas) {Canvas\\(sprites, mapa, animações)};
    \node[box, right=3.2cm of canvas] (html) {HTML semântico\\(texto, botões, descrições)};

    \node[box, below=1.6cm of $(canvas)!0.5!(html)$, minimum width=9.8cm] (ui)
      {Interface final no navegador\\(camada visual + camada informativa)};

    \draw[->] (canvas) -- (ui);
    \draw[->] (html) -- (ui);

  \end{tikzpicture}
  \caption{Separação entre a camada visual (canvas) e a camada informativa 
  (HTML semântico), permitindo acessibilidade e melhor suporte a tecnologias 
  de apoio.}
  \label{fig:canvas-html-semantico}
\end{figure}

\subsubsection{Independência de plataforma e performance}

\subsection{Planeamento e organização do trabalho}

% todo separar conceito de regras

\subsection{Loop de gameplay}
%exploração → sair → quiz (6) → reinício

\subsection{Regras e Progressão}
% todo atualizar
\begin{itemize}
  \item Estrutura de Níveis: O jogo será dividido em níveis (ex:
    cozinha, sala, exterior da casa).
  \item Desbloqueio de Níveis: A regra principal de progressão é que
    o jogador só poderá avançar para o nível seguinte após ter
    completado com sucesso o nível anterior.
  \item Validação da Aprendizagem: A conclusão de um nível será
    validada através de um minijogo, especificamente um questionário
    sobre os minerais e informações exploradas nessa área. A
    aprovação neste questionário é o que permite o desbloqueio do próximo nível.
  \item Sistema de Pontuação (Opcional): Está a ser ponderada a
    inclusão de um sistema de pontuação, que poderá ser influenciado
    pelo desempenho no questionário. Uma ideia em discussão é a
    inclusão de um temporizador por nível; um tempo de conclusão mais
    rápido poderia resultar numa pontuação mais alta, incentivando a
    rejogabilidade.
\end{itemize}

\subsection{Âmbito do Jogo}

O âmbito do jogo foca-se na experiência de descoberta dentro deste
``mundo'' virtual. O jogador não tem ``regras'' de exploração; pode
clicar em qualquer objeto, em qualquer ordem, dentro do nível ativo.
A ``Casa dos Minerais'' serve, assim, como o hub central para esta descoberta.

\subsection{Como integrar outros públicos?}
\label{sec:other}

Os problemas falados em \ref{sec:forwho}, já foram todos pensados, e já
existem soluções para alguns dos problemas. Nós só necessitamos
implementar essas soluções.

\subsubsection{Acessibilidade}

No que toca a Web, existe a \href{https://www.w3.org/WAI/}{Web
Accessibility Initiative (WAI)} do \href{https://www.w3.org/}{World
Wide Web Consortium (W3C)}. Criaram um conjunto normas para criar
um Web acessível, e nos só temos de seguir as normas: HTML
semântico e ARIA; texto para leitores de ecrã; texto simples;
imagens e vídeos (média) acessíveis; controlos de interface
semânticos; cores e contraste; ...

\subsubsection{Internacionalização}

Aprender não esta reservado para uma só cultura, região, língua,
etnia... A W3C também esta preocupada neste aspeto, e vemos isso
por exemplo com a
\href{https://developer.mozilla.org/en-US/docs/Web/JavaScript/Reference/Global_Objects/Intl}{Intl
Web API}. Não somos poliglotas, mas com certeza encontraremos forma
de permitir com que este projeto não seja apenas em português e
inglês. Os browsers normalmente conseguem traduzir um website, mas
se não quisermos depender disso, podemos explorar opções como
\href{https://github.com/argosopentech/argos-translate}{Argos}.

\subsubsection{Informação semântica (HTML vs canvas)}

O jogo quando jogado no navegador, não podemos depender só do HTML
`canvas` para mostrar a informação, porque a informação no `canvas`
não passa de pixéis, de bits num bitmap (um leitor de ecrãs não
consegue ler texto no `canvas`). Devemos usar HTML semântico para
qualquer informação que seja de valor.

\subsubsection{Independência de plataforma e performance}

O jogo tem de ser possível de ser jogado na maioria dos
dispositivos, e não só no nosso computador. Existem pessoas que
talvez apenas tenham um smartphone, e que nem é dos “bons”. Temos
dois problemas aqui então. Para o primeiro, a solução vai ser não
criar o jogo apenas para navegador web através da biblioteca
\href{https://www.raylib.com/index.html}{raylib}, e então o jogo
poderá rodar nativamente num PC (caso o PC não tenha suporte para
os browsers mais atualizados) independentemente do sistema
operativo, num Android, numa PlayStation da Sony ou numa Switch da
Nintendo. Para o segundo problema, a performance: tentar escrever
código otimizado, a pensar que existem computadores mais fraquitos;
controlar a quantidade e o tamanho das requisições, a pensar nos
sítios onde a internet e mais lenta; caches; armazenamento local; o
offline. O jogo não vai ser só para os computadores mais topo de
gama. Não vais ter de comprar um novo dispositivo para jogar. Uma
decisão ecológica e amiga do ambiente, mas também amiga da tua carteira!

\subsection{Planeamento e organização do trabalho}

\section{Tech Stack}

\subsection{Linguagem (Odin)}
\subsection{Biblioteca gráfica/multimédia (raylib)}
\subsection{Build web / WebAssembly (WASM)}
\subsection{Scripts/ferramentas de build e colaboração (Git, etc.)}

%%%

% \begin{figure}[!htb]
%   \centering
%   \caption{Exemplo de Figura flutuante (\textit{in}
%   \url{http://en.wikipedia.org/wiki/Class_diagram})}
%   \includegraphics[width=13cm]{exemploFig}
%   \label{fig:exemplofig}
% \end{figure}
%
% \begin{center}
%
%   \captionof{figure}{Exemplo de Figura não flutuante (\textit{in}
%   \url{http://en.wikipedia.org/wiki/Class_diagram})}
%   \includegraphics[width=\textwidth]{exemploFig}
%   %\includegraphics[width=0.5\textwidth]{exemploFig}
%   \label{fig:exemplofigFixa}
% \end{center}

% \lstinputlisting[language=java,abovecaptionskip=20pt,caption={Exemplo
%     de listagem de parte de um ficheiro Java na pasta
% listagens.},{label=lst:javaparcial},firstline=38,lastline=44]{listagens/quicksort.java}

% \noindent
% \begin{minipage}{\linewidth}
%   \captionof{table}{Uma tabela de exemplo não flutuante} % título da tabela
%   \begin{tabular}{l l c r} % duas colunas à esquerda (l l), uma ao
%     % centro (c) e uma à direita (r) (4 colunas)
%     \hline\hline %insere duas linhas horizontais
%     Caso & Método 1 & Método 2 & Método 3 \\ [0.5ex] % insere tabela
%     \hline % insere uma linha horizontal
%     1 & 50 & 837 & 970 \\ % insure corpo da tabela
%     2 & 47 & 877 & 230 \\
%     3 & 31 & 25 & 415 \\
%     4 & 35 & 144 & 2356 \\
%     5 & 45 & 300 & 556 \\ [1ex] % [1ex] adiciona espaço vertical
%     \hline %insere uma linha horizontal
%   \end{tabular}
%   \label{tab:casoFIXO} % é utilizada para referir a tabela no texto
%   \\ %% linha em braco
%   \textit{Nota}: uma nota. remover se não for necessária.\\
% \end{minipage}
%
% %baseado num exemplo em http://www1.maths.leeds.ac.uk/latex/TableHelp1.pdf
% \begin{table}[!htb]
%   \caption{Uma tabela de exemplo}
%   \begin{tabular}{l l c r} % duas colunas à esquerda (l l), uma ao
%     % centro (c) e uma à direita (r) (4 colunas)
%     \hline\hline %insere duas linhas horizontais
%     Caso & Método 1 & Método 2 & Método 3 \\ [0.5ex] % insere tabela
%     \hline % insere uma linha horizontal
%     1 & 50 & 837 & 970 \\ % insere corpo da tabela
%     2 & 47 & 877 & 230 \\
%     3 & 31 & 25 & 415 \\
%     4 & 35 & 144 & 2356 \\
%     5 & 45 & 300 & 556 \\ [1ex] % [1ex] adiciona espaço vertical
%     \hline %insere uma linha horizontal
%   \end{tabular}
%   \label{tab:caso} % é utilizada para referir a tabela no texto
%   \\ %% linha em braco
%   \textit{Nota}: uma nota. remover se não for necessária.\\
% \end{table}
