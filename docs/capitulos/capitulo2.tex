\chapter{Desenvolvimento}
\label{ch:develop}

\section{Enquadramento teórico}

\subsection{``Estado da arte''}

«Inspirado por jogos educativos como \textcite{briefcasegame},
[...] o objetivo é aumentar a literacia sobre recursos
geológicos, demonstrando de forma lúdica e visual a ligação direta
entre a geologia e a nossa vida diária.»

% O nosso toque pessoal, point click tipo Club Penguin...

\subsection{Os minerais}
\label{sec:minerals}

Eles andam por aí espalhados, mas tu nem pensas neles: os minerais.
Matéria-prima e recurso importante para diversos setores como a
indústria química, da construção civil, da agricultura e da
energia. Mas é que estão presentes em todo o lado mesmo:

\begin{itemize}
  \item Se estás a ler isto através de um aparelho eletrónico,
    esquece, desde o cobre e ouro dos cabos elétricos até ao
    silício dos microchips;
  \item Mas mesmo lendo numa folha de papel, é capaz dessa folha
    ter vestígios de cádmio, cobalto e titânio.
\end{itemize}

As frutas e vegetais que consomes, além das vitaminas, também têm os
minerais presentes no solo. Minerais são indispensáveis para a
nossa vida e para a sociedade.

O objetivo principal deste jogo é ensinar onde é que estes
minerais, que existem naturalmente no nosso planeta, estão
presentes nas coisas que usamos todos os dias. Além isso,
ensinamos qual a utilidade de certo mineral numa ocasião específica.
Mas basta isso?

Uma das razões para a qual estes minerais ficam escondidos dos
consumidores é porque, tu vais à loja e compras o telemóvel, mas o
processo de produção desse telemóvel, da extração dos minerais da
terra, da montagem do equipamento, até ao produto final, é
totalmente ofuscado. Existe uma alienação entre as nossas vidas e a
geologia — os minerais — e todo o trabalho humano que leva à
realização dos teus bens. Simplesmente o telemóvel não apareceu na
estante da loja por pozinhos de perlimpimpim. O telemóvel é visto
como uma coisa e não como fruto de mineração e do trabalho. Tudo
isto já havia sido dito na \ref{ch:intro}.

Usamos o jogo para conectar também o mineral, ao país de origem
(geografia), e ao trabalho e/ou processo necessário para esse
material estar disponível.

\subsection{Gamificação na pedagogia}

A ``Casa dos Minerais'' utilizará elementos de gamificação para
estruturar a aprendizagem, como a progressão por diferentes níveis
e a inclusão de minijogos, por exemplo, um questionário no final de
cada nível.

No entanto, a abordagem pedagógica deste projeto foca-se em usar a
brincadeira como uma forma de descoberta e reflexão. O objetivo
principal não é criar um sistema de avaliação rigoroso; o foco é a
alegria de aprender. A aprendizagem é mais importante do que uma
pontuação final ou a rapidez com que se termina o jogo.

Desta forma, elementos como a pontuação ou um temporizador,
servirão como um incentivo opcional (conforme detalhado na secção
\ref{sec:future}) e não como o foco central da experiência.

\subsection{Para quem é o jogo?}
\label{sec:forwho}

O público-alvo principal são crianças e jovens, servindo como ferramenta de apoio às disciplinas de Estudo do Meio e Ciências Naturais (no contexto curricular português). No entanto, o design visual e a mecânica de exploração foram pensados para serem apelativos a qualquer faixa etária, desde que exista curiosidade em compreender a origem geológica dos objetos do nosso quotidiano. O jogo procura, assim, ser uma experiência transversal, capaz de captar o interesse tanto de alunos como de adultos.

Idealmente, o acesso ao conhecimento deve ser universal, independentemente de limitações físicas, cognitivas ou sensoriais. Reconhecemos que a verdadeira inclusão exige cuidados específicos, como o suporte a navegação por teclado, compatibilidade com leitores de ecrã e ajustes de contraste ou tipografia. Contudo, devido ao âmbito temporal e académico deste projeto, o protótipo atual foca-se na implementação das mecânicas base e na validação do conceito pedagógico, estando a interação limitada ao uso do rato e à interface visual padrão.

Desta forma, embora a versão atual esteja funcional para o público geral, a implementação plena de normas de acessibilidade (WCAG) permanece como uma prioridade identificada para iterações futuras, de modo a cumprir a visão de um jogo efetivamente para todos.

\subsection{Aprendizagem por descoberta e contextualização}
%Como “ancorar” conceitos abstratos (minerais) em exemplos concretos (objetos em casa/escola). Dá base teórica para o vosso design de cenário
A aprendizagem por descoberta baseia-se na ideia de que o aluno compreende 
melhor quando participa ativamente no processo, explorando, 
pensando por ele mesmo e construindo ligações por iniciativa própria. 
Em vez de receber a informação de forma expositiva, 
o utilizador é colocado num ambiente interativo onde a curiosidade 
e a exploração conduzem naturalmente à aprendizagem.

No caso do Casa dos Minerais, 
esta abordagem é reforçada pela contextualização: conceitos abstratos como “mineral”, 
“recurso” ou “matéria-prima” são ancorados em exemplos 
concretos e familiares, como objetos existentes numa casa ou numa escola. 
Ao interagir com esses objetos, o jogador passa a associar materiais 
e tecnologias do quotidiano à sua origem geológica, 
tornando o conteúdo mais acessível e memorável. 
Esta ligação direta entre o que o utilizador já reconhece 
(um frigorífico, uma torradeira, um computador) e o 
conceito que se pretende ensinar (minerais e o seu papel no fabrico) 
reduz a distância entre teoria e prática.

\subsection{Ciclo pedagógico e questionários como reforço}

O ciclo do jogo combina aprendizagem por descoberta 
com reforço através de avaliação formativa. 
Numa primeira fase, o utilizador explora o cenário, 
interage com objetos do quotidiano e consulta a informação 
sobre os minerais associados. Esta exploração permite 
compreender os conceitos no respetivo contexto.

Numa segunda fase, o utilizador inicia um questionário curto 
para recuperar ativamente o que observou durante a exploração. 
Este processo de recordação ativa reforça a retenção 
e funciona como mecanismo de feedback, ao tornar visível 
o que foi compreendido e o que precisa de revisão.

No final, o ciclo pode ser repetido: o utilizador regressa 
ao cenário, revê conteúdos e volta a testar conhecimentos. 
Esta alternância entre descoberta, verificação e repetição 
suporta aprendizagem progressiva em sessões curtas.
\subsection{Qualidade do conteúdo e fontes}
%Contraste, tamanho de texto, linguagem simples, input por rato/teclado. Mesmo que implementem pouco, dá para discutir “boas práticas”.

A qualidade do conteúdo é fundamental para garantir que o jogo seja acessível,
envolvente e educativo. Para conseguir isso, é importante seguir boas práticas
e apresentação de informação, como:

\begin{itemize}
  \item Fontes fiáveis: as associações entre objetos do quotidiano e minerais
  devem ser baseadas em fontes reconhecidas (por exemplo, recursos educativos oficiais,
  instituições científicas ou bases de dados geológicas), evitando informação não verificada.
  Neste trabalho foram privilegiadas fontes institucionais e educativas como a USGS
  \textcite{usgs_education_mrp,usgs_minerals101} e a Minerals Education Coalition
  \textcite{mec_mineralsinyourlife}.

  \item Confirmação e coerência: sempre que possível, a informação é cruzada entre mais do que uma referência,
  reduzindo erros e ambiguidades.

  \item Consistência terminológica: utilização consistente dos termos (por exemplo, “mineral” vs “minério”),
  nomes dos minerais, e forma de escrita ao longo do jogo, para evitar confusão no utilizador.

  \item Estrutura uniforme do conteúdo: manter um formato semelhante nas descrições (ex.: mineral associado,
  uso no objeto e país de origem), facilitando a leitura e a comparação entre objetos.

  \item Linguagem adequada ao público-alvo: textos claros e diretos, com o mínimo de palavreado técnico,
  para garantir compreensão por crianças e jovens, sem perder rigor.
\end{itemize}


\section{Metodologia}

Para a implementação, o projeto baseia-se em tecnologias de multiplataforma, tirando partido da linguagem de programação Odin. Para a parte gráfica escolhemos a Raylib, integrada com Odin. Esta combinação permite criar gráficos 2D interativos com desempenho nativo e exportação para WebAssembly (WASM). A visualização final no navegador é assegurada pela integração deste módulo WASM com tecnologias web padrão (HTML5, CSS3 e JavaScript), garantindo compatibilidade e acesso universal.

\subsection{Identificação dos minerais usados no quotidiano}

Minerais são todas as substâncias naturais formadas por processos
geológicos que normalmente consegues encontrar na crosta da terra.

A telhado das casas podem ser feitos com argila, os vidros com areia.
Na cozinha, tens pratos que podem conter areia, calcário, feldspato e
argila. O sal vem do sal-gema. Uma panela pode ser de ferro, níquel e
crómio. O frigorifico pode ser alumínio e cobre, e derivados de
petróleo. Na cozinha também pode ser usado gás natural. Existem
fotografias que usam prata e bronze. Um vaso de flores pode ser de
argila. Um livro pode ter vestígios de cadmio, cobalto e titânio. Uma
lâmpada pode ter níquel, alumínio e chumbo. Um relógio pode ter
silício, crómio, ferro e níquel. Um espelho terá quartzo, alumínio e
prata. O papel higiénico gesso e calcário. A pasta de dentes tem fluorite.

O mais impressionante talvez são os bens tecnológicos como
televisões/monitores, computadores e telemóveis, tão presentes no
nosso dia"-a"-dia e que os levamos de um lado para o outro connosco.
Estes podem conter zinco, crómio, silício, alumínio, cobre, estanho,
manganês, cobalto, bário, platina, tungsténio, arsénio, lítio, ouro, prata, ...

\subsection{Critérios de seleção de objetos (casa/escola)}

A seleção de objetos para os níveis ``casa'' e ``escola'' foi guiada por critérios 
pedagógicos e práticos, de forma a garantir que o conteúdo 
fosse relevante, compreensível e adequado ao formato do jogo. 
Em primeiro lugar, foram priorizados objetos comuns e facilmente 
reconhecíveis, pois isso reduz a barreira de entrada e ajuda o 
utilizador a relacionar o conteúdo com a sua experiência do dia a dia. 
Em segundo lugar, deu-se preferência a objetos que permitissem associações 
claras com minerais/minérios utilizados no seu fabrico, 
garantindo que cada interação tivesse valor educativo. 

Adicionalmente, procurou-se diversidade de categorias (por exemplo, cozinha, 
lavandaria, eletrónica e música), evitando repetição e permitindo abordar 
diferentes utilizações de recursos minerais. Do ponto de vista do design do jogo, 
os objetos selecionados também tiveram de ser representáveis visualmente 
e inseridos no cenário de forma coerente, com áreas de interação claras 
e sem sobrecarregar o mapa com demasiados pontos clicáveis. 

Por fim, a existência de dois contextos (casa e escola) permite variar o 
conteúdo mantendo o mesmo fluxo de interação. O nível da casa favorece 
objetos ligados à habitação e consumo doméstico, enquanto o nível da escola 
introduz elementos típicos do ambiente escolar. Esta progressão por contexto 
facilita aprendizagem comparativa e incentiva o utilizador a explorar 
novamente para descobrir novas associações entre objetos e minerais.

Para suportar esta seleção de forma consistente, os objetos e minerais foram 
representados em dados estruturados, permitindo reutilização e extensão do 
conteúdo sem alterar a lógica do jogo. Cada objeto inclui um nome, uma 
categoria e uma lista de minerais/minérios associados, onde cada mineral 
regista a sua origem e a função no objeto. Esta estrutura facilita a 
manutenção do conteúdo e permite escalar o projeto para novos cenários 
(como o nível de escola), mantendo o mesmo formato. 

\begin{lstlisting}[language=json, caption={Excerto do dataset de objetos e minerais.}, label={lst:dataset-objetos}]
[
  {
    "name": "Fogão Elétrico",
    "category": "Cozinha",
    "minerals": [
      {
        "name": "Ferro",
        "origin": "Austrália, Brasil, China",
        "use": "Corpo do fogão e resistências"
      },
      {
        "name": "Crómio",
        "origin": "África do Sul, Cazaquistão",
        "use": "Resistências"
      },
      {
        "name": "Níquel",
        "origin": "Indonésia, Filipinas, Rússia",
        "use": "Revestimento"
      }
    ]
  }
]
\end{lstlisting}

\subsection{Comparação com outros jogos}

Em comparação com o \textcite{briefcasegame}, o \textit{Casa dos Minerais} 
aposta numa abordagem mais exploratória. O jogador percorre um cenário 
(point-and-click) e encontra objetos distribuídos no ambiente, 
descobrindo informação associada a cada interação. No \textit{BriefCase game}, 
a dinâmica é mais direta e centrada na associação imediata entre objeto 
e mineral, com menor ênfase na exploração espacial e no contexto do objeto. 
Assim, o nosso jogo privilegia a contextualização (objeto no seu ``habitat'') 
antes do momento de verificação através do questionário.

Relativamente ao \textcite{min4kids-minerals-house}, existe uma proximidade 
no objetivo educativo (relacionar minerais com objetos do quotidiano), 
mas a apresentação difere. Esse recurso tende a ser mais informacional, 
funcionando como consulta/explicação, enquanto o \textit{Casa dos Minerais} 
procura integrar a informação num fluxo de jogo, combinando descoberta no mapa 
com um quiz de escolha múltipla (4 opções, 1 correta) para reforço do conteúdo. 
Desta forma, o utilizador alterna entre explorar e testar o que aprendeu.

No caso do \textcite{wordwall-rochas-minerais}, o foco está num conjunto de 
atividades curtas e simples, normalmente com interações rápidas e regras 
minimalistas. Embora sejam úteis como exercícios imediatos, tendem a ter menos 
camada de contexto e menos continuidade. Em contraste, o nosso projeto mantém 
um cenário persistente, permitindo aprender por exploração e regressar ao mapa 
após o questionário para rever ou descobrir novos objetos.

Em síntese, o \textit{Casa dos Minerais} distingue-se por juntar:
\begin{enumerate}
  \item exploração contextualizada em cenários familiares; e
  \item validação por questionário curto, suportando a repetição do ciclo para
  consolidação progressiva da aprendizagem.
\end{enumerate}



\begin{table}[H]
\centering
\caption{Comparação entre recursos/jogos educativos sobre minerais.}
\label{tab:comparacao-jogos}
\begin{tabular}{p{3.2cm} p{3.6cm} p{4.0cm} p{4.0cm}}
\hline
\textbf{Recurso} & \textbf{Tipo de interação} & \textbf{Contexto (cenário)} & \textbf{Avaliação / feedback} \\
\hline
\textcite{briefcasegame} 
& Associação direta objeto--mineral 
& Pouco contexto espacial; foco na correspondência 
& Feedback imediato da associação \\
\hline
\textcite{min4kids-minerals-house} 
& Consulta/informação com interação limitada 
& Conteúdo mais informacional; menor componente de ``jogo'' 
& Sem validação estruturada; reforço principalmente pela leitura \\
\hline
\textcite{wordwall-rochas-minerais} 
& Mini-jogos simples (atividades curtas) 
& Contexto reduzido; tarefas isoladas 
& Feedback rápido por exercício, geralmente sem progressão contínua \\
\hline
\textit{Casa dos Minerais} 
& Exploração point-and-click + quiz de escolha múltipla 
& Cenários familiares (casa/escola) com objetos distribuídos 
& Quiz curto (6 perguntas, 4 opções, 1 correta) e possibilidade de voltar a explorar \\
\hline
\end{tabular}
\end{table}


\subsection{Diferenças de abordagem (exploração vs associação direta)}

Uma diferença central entre jogos educativos sobre minerais está na forma como 
o conhecimento é apresentado e praticado. Numa abordagem de \textbf{exploração}, 
o utilizador aprende ao interagir com um ambiente e ao descobrir conteúdo no 
contexto em que os objetos aparecem. Isto favorece a contextualização, pois o 
objeto não surge isolado: está inserido num cenário familiar (por exemplo, uma 
casa ou uma escola), o que facilita a ligação entre o conceito abstrato 
(mineral/minério) e o exemplo concreto (objeto do quotidiano).

Em contraste, numa abordagem de \textbf{associação direta}, o objetivo é mais 
imediato: o utilizador recebe um objeto (ou um nome/imagem) e deve associá-lo 
ao mineral correto. Este formato tende a ser mais rápido e focado em treino, 
mas pode reduzir a componente de descoberta e o contexto de uso do objeto, 
dependendo da implementação.

No \textit{Casa dos Minerais}, a exploração funciona como fase inicial de 
aprendizagem, permitindo observar e compreender as associações objeto--mineral 
antes da validação. Em seguida, o questionário de escolha múltipla (4 opções, 
1 correta) força a recordação ativa dessas associações, combinando os benefícios 
de ambos os estilos: primeiro contextualiza-se, depois pratica-se e verifica-se.

\subsection{Limitações observadas e decisões adotadas no projeto}

Durante a análise de recursos e ao longo do desenvolvimento, 
foram identificadas limitações comuns em abordagens educativas sobre minerais. 
Uma delas é a tendência para conteúdos demasiado expositivos, onde o utilizador 
lê informação mas tem pouca oportunidade de a aplicar. Outra limitação é a 
associação direta sem contexto, que pode treinar a memorização de pares 
objeto--mineral, mas nem sempre ajuda a compreender a relevância desses minerais 
no quotidiano.

Para responder a estas limitações, o \textit{Casa dos Minerais} adotou decisões 
de design que equilibram simplicidade com contexto. Em vez de depender apenas 
de exercícios isolados, o jogo organiza a aprendizagem em dois momentos: 
(1) exploração num cenário familiar, onde a informação surge ligada a objetos 
do dia a dia; e (2) um questionário curto de escolha múltipla para reforço e 
validação, mantendo o ritmo e evitando fadiga.

Do ponto de vista visual, também existiram limitações na produção de assets. 
Numa fase inicial, foi testada a geração de elementos gráficos com o Gemini, 
mas os resultados não cumpriam requisitos técnicos essenciais para um mapa 
baseado em tiles, nomeadamente a disponibilização de um \textit{tileset} com 
grelha 32$\times$32\,px. Apesar de visualmente apelativas, as imagens geradas 
surgiam tipicamente em dimensões muito grandes e sem organização reutilizável, 
o que dificultava recorte, alinhamento e integração consistente no cenário. 
Por esse motivo, optou-se por adquirir um conjunto de assets estruturados, 
garantindo compatibilidade com o formato do mapa, consistência visual e uma 
experiência mais polida.

% todo meter melhores imagens com tamanhos parecidos para nao ficar aquele bloco 
\begin{figure}[H]
  \centering
  \begin{subfigure}[t]{0.58\textwidth}
    \centering
    \includegraphics[width=\textwidth]{figuras/gemini_assets.jpeg}
    \caption{Asset gerado automaticamente (Gemini): imagem visualmente apelativa, 
    mas sem estrutura de \textit{tileset} e sem grelha 32$\times$32\,px, 
    dificultando recorte, alinhamento e reutilização no mapa.}
    \label{fig:gemini-gerado}
  \end{subfigure}
  \hfill
  \begin{subfigure}[t]{0.38\textwidth}
    \centering
    \includegraphics[width=\textwidth]{figuras/paid_assets.png}
    \caption{Tileset estruturado: elementos organizados e dimensionados, 
    facilitando integração, consistência visual e reutilização no cenário.}
    \label{fig:tileset-estruturado}
  \end{subfigure}

  \caption{Comparação entre geração automática e assets estruturados para jogos 2D.
  Apesar da qualidade estética, a ausência de formato e escala (ex.: 32$\times$32\,px) 
  na geração automática limitou a utilização prática no projeto.}
  \label{fig:comparacao-gemini-tileset}
\end{figure}

A Figura~\ref{fig:comparacao-gemini-tileset} ilustra que a principal limitação 
não foi estética, mas sim técnica: o projeto exigia assets reutilizáveis em 
tilesets com dimensões uniformes (32$\times$32\,px) para montagem eficiente do mapa.

Em contrapartida, a utilização de IA (Gemini) revelou-se bastante útil para a criação de elementos individuais, nomeadamente as representações em \textit{pixel art} dos próprios minerais. Ao contrário dos \textit{tilesets} de mapa que exigem encaixe perfeito, as imagens isoladas foram geradas com qualidade satisfatória, necessitando apenas de pequenas correções pontuais (como limpeza de fundo ou ajustes de detalhes) realizadas em software de edição de imagem como o Photopea ou Photoshop.

\begin{figure}[H]
  \centering
  \begin{subfigure}[t]{0.3\textwidth}
    \centering
    \frame{\includegraphics[width=0.6\textwidth]{figuras/mineral_ouro.png}}
    \caption{Ouro}
  \end{subfigure}
  \hfill
  \begin{subfigure}[t]{0.3\textwidth}
    \centering
    \frame{\includegraphics[width=0.6\textwidth]{figuras/mineral_litio.png}}
    \caption{Lítio}
  \end{subfigure}
  \hfill
  \begin{subfigure}[t]{0.3\textwidth}
    \centering
   \frame{ \includegraphics[width=0.6\textwidth]{figuras/mineral_cobre.png}}
    \caption{Cobre}
  \end{subfigure}
  \caption{Exemplos de \textit{pixel art} de minerais gerados por IA (Gemini) e refinados manualmente.}
  \label{fig:minerais-ia}
\end{figure}

Para obter estes resultados, foi necessário especificar características detalhadas no \textit{prompt} (instrução dada à IA), como o estilo visual, a perspetiva isométrica e a paleta de cores. Um exemplo do \textit{prompt} utilizado para gerar o ícone de ``Terras Raras'' foi:

\begin{quote}
``Terras Raras Pixel art icon of rare earth minerals, 32x32 pixels, multicolored iridescent crystalline rock cluster with rainbow shimmer effect (mix of green, blue, purple, pink hues), dark outlines, simple geometric facets showing 3D depth, clean retro RPG game item style matching these references, transparent background, isometric 3/4 view, magical glowing appearance''
\end{quote}

Outra decisão importante foi manter o fluxo de interação simples e repetível. 
O utilizador pode regressar ao mapa após o questionário para rever objetos já 
descobertos ou encontrar novos, permitindo aprendizagem progressiva sem impor 
uma sequência rígida. Por fim, ao estruturar os dados de conteúdo (objetos, 
minerais, origem e utilização) de forma consistente, o projeto facilita a 
manutenção e a expansão para novos cenários, como a introdução de um nível 
centrado numa escola.

\subsection{Conceito, regras e âmbito do jogo}

\subsubsection{Conceito}

O conceito central do ``Casa dos Minerais'' foca-se na exploração 
interativa e autoguiada de um ambiente virtual, começando numa casa, 
e podendo ser expandido para outros cenários, como uma escola. 
Inspirado num formato \textit{point-and-click}, semelhante ao ``Club 
Penguin'', o jogador tem liberdade para navegar pelo mapa e interagir 
com objetos do quotidiano.

Ao clicar em objetos comuns (por exemplo, um frigorífico, uma televisão 
ou uma torradeira), o jogador acede a informação sobre os minerais 
associados ao seu fabrico. O objetivo é tornar visível a relação entre 
recursos naturais e produtos do dia a dia, através de exemplos concretos 
e familiares.

\subsubsection{Loop de gameplay}

O \textit{Casa dos Minerais} segue um ciclo simples e repetível, 
pensado para alternar descoberta e reforço do conteúdo. 
O jogador inicia num cenário (por exemplo, a casa) e explora o mapa, 
interagindo com objetos do quotidiano para desbloquear informação 
sobre os minerais/minérios associados.

Quando decide consolidar o que aprendeu, o jogador sai do cenário 
e inicia o questionário ao clicar numa área específica (marcada pela cor azul). 
O quiz é composto por 6 perguntas de escolha múltipla, com 4 opções 
por pergunta e apenas 1 resposta correta.

\begin{figure}[H]
  \centering
  \includegraphics[width=0.55\textwidth]{figuras/quiz_area.png}
  \caption{Zona de transição onde o questionário é iniciado. 
  Ao sair do mapa e clicar no ``chão azul'', começa o quiz.}
  \label{fig:chao-azul-quiz}
\end{figure}

Para avançar para o cenário seguinte, o jogador deve obter pelo menos 
50\% de respostas corretas no questionário. Caso não atinja esse valor, 
pode regressar ao mapa e tentar novamente após rever os conteúdos.

Após concluir o quiz, o jogador regressa ao mapa, podendo continuar a 
exploração no mesmo cenário ou prosseguir para o próximo, caso cumpra 
o critério de aprovação. Este ciclo promove aprendizagem progressiva, 
mantendo sessões curtas e um fluxo consistente entre cenários.

\subsubsection{Regras e progressão}

As regras do jogo privilegiam clareza e acessibilidade. A exploração 
não impõe uma sequência fixa de objetos nem penaliza o jogador por 
escolhas de percurso, permitindo diferentes estratégias de descoberta.

A progressão entre cenários é condicionada pelo desempenho no momento 
de consolidação (questionário). Quando o jogador não atinge o critério 
mínimo, a experiência incentiva a revisão do conteúdo através do regresso 
ao mapa, promovendo repetição e melhoria incremental.

\subsubsection{Âmbito do jogo}

O âmbito do \textit{Casa dos Minerais} centra-se na descoberta de informação 
sobre minerais/minérios em objetos do quotidiano e no reforço desse conteúdo 
através de um mecanismo simples de questionário. O foco do projeto é oferecer 
uma experiência curta, clara e educativa, adequada para uso autónomo ou em 
contexto de sala de aula.

Não fazem parte do âmbito atual funcionalidades avançadas como inventário, 
sistemas complexos de personalização, narrativa extensa, economia de jogo, 
ou modos competitivos. A prioridade é a qualidade do conteúdo, a coerência 
visual e a estabilidade da experiência em diferentes dispositivos.


\subsection{Como integrar outros públicos?}
\label{sec:other}

\subsubsection{Acessibilidade}

Para que o jogo seja útil a diferentes públicos, a acessibilidade deve ser 
considerada como parte do design e não apenas como um extra. Em contexto web, 
existem recomendações consolidadas pelo W3C através da \textit{Web Accessibility 
Initiative (WAI)}, que servem como referência para tornar interfaces mais 
inclusivas.

Como o jogo é renderizado num \textit{canvas}, é importante garantir que a 
informação relevante (por exemplo, textos descritivos e botões/ações) não 
depende exclusivamente de pixels no ecrã. Sempre que possível, essa informação 
deve existir também em elementos HTML semânticos, permitindo leitura por 
tecnologias de apoio, como leitores de ecrã.

Do ponto de vista de interface, boas práticas incluem: contraste adequado entre 
texto e fundo, tamanhos de letra legíveis, linguagem clara e consistente, e 
feedback visual percetível quando ocorre uma interação. Atualmente, a interação 
do jogo é feita apenas com o rato, pelo que a expansão para suporte de teclado e 
toque (\textit{touch}) constitui uma melhoria relevante para aumentar a 
acessibilidade em diferentes dispositivos.

\subsubsection{Internacionalização}

A internacionalização é importante para que o jogo possa ser utilizado por 
públicos com diferentes línguas e contextos culturais, sem alterar a lógica do 
programa. A estratégia adotada passa por separar texto e conteúdo localizável 
do código, mantendo strings de interface e textos informativos (nomes de 
objetos, minerais e descrições) em ficheiros externos, selecionados consoante 
o idioma.

Esta abordagem facilita suportar, no mínimo, português e inglês, e permite 
adicionar novos idiomas no futuro com impacto reduzido na implementação. 
Embora os browsers disponibilizem tradução automática, a localização nativa 
com recursos próprios oferece maior controlo terminológico e maior consistência 
do conteúdo educativo.

\subsubsection{Informação semântica (HTML vs canvas)}

Quando o jogo é executado no navegador, grande parte do conteúdo visual é 
renderizado num elemento HTML \texttt{canvas}. No entanto, o \texttt{canvas} 
representa a informação como pixels, o que significa que tecnologias de apoio 
(como leitores de ecrã) não conseguem interpretar diretamente o texto ou a 
estrutura apresentada no ecrã.

Por este motivo, sempre que existe informação relevante para a aprendizagem 
(por exemplo, descrições de minerais, botões de interface ou mensagens de 
feedback), é preferível disponibilizá-la também através de HTML semântico, 
com elementos apropriados. Esta abordagem 
melhora a acessibilidade, facilita indexação e permite maior compatibilidade 
com ferramentas de tradução e de apoio ao utilizador.

No contexto deste projeto, esta distinção é particularmente importante porque 
o conteúdo educativo não deve depender apenas da representação gráfica. Ao 
separar a camada visual (canvas) da camada informativa (HTML semântico), 
torna-se possível manter a experiência de jogo e, ao mesmo tempo, aumentar a 
inclusão e a reutilização do conteúdo.

\begin{figure}[H]
  \centering
  \begin{tikzpicture}[
    node distance=1.2cm,
    every node/.style={align=center},
    box/.style={draw, rounded corners, minimum width=4.6cm, minimum height=1.1cm}
  ]

    \node[box] (canvas) {Canvas\\(sprites, mapa, animações)};
    \node[box, right=3.2cm of canvas] (html) {HTML semântico\\(texto, botões, descrições)};

    \node[box, below=1.6cm of $(canvas)!0.5!(html)$, minimum width=9.8cm] (ui)
      {Interface final no navegador\\(camada visual + camada informativa)};

    \draw[->] (canvas) -- (ui);
    \draw[->] (html) -- (ui);

  \end{tikzpicture}
  \caption{Separação entre a camada visual (canvas) e a camada informativa 
  (HTML semântico), permitindo acessibilidade e melhor suporte a tecnologias 
  de apoio.}
  \label{fig:canvas-html-semantico}
\end{figure}

\subsubsection{Independência de plataforma e performance}

Um objetivo importante do projeto é garantir que o jogo possa ser utilizado em 
diferentes plataformas e em dispositivos com capacidades variadas. Em contexto 
educativo, é comum existirem restrições de hardware e de rede, pelo que uma 
solução multiplataforma reduz barreiras de acesso e facilita adoção em sala de 
aula ou em uso autónomo.

Atualmente, o jogo é disponibilizado no navegador, permitindo acesso imediato 
sem instalação. No entanto, a base de código foi estruturada de forma a manter 
a lógica do jogo separada de detalhes específicos da plataforma, tornando mais 
simples uma futura adaptação para execução nativa noutros sistemas. Esta opção 
preserva o investimento de desenvolvimento e facilita a expansão do projeto.

Do ponto de vista de performance, a execução em browser exige cuidados 
particulares, como reduzir o peso de assets, evitar carregamentos desnecessários 
e manter um \textit{loop} de atualização/renderização eficiente. 
Em geral, a performance depende sobretudo de:
\begin{itemize}
  \item gestão de recursos gráficos (tamanho e número de sprites/tiles);
  \item otimização do ciclo de jogo (\textit{update}/\textit{draw});
  \item minimizar operações custosas durante a interação.
\end{itemize}
Estas medidas ajudam a garantir uma experiência fluida e consistente, 
mesmo em dispositivos mais modestos.

\subsection{Planeamento e organização do trabalho}
Numa fase inicial, o planeamento do projeto foi realizado em conjunto, 
com foco na definição do mapa, do fluxo do jogo e das principais 
funcionalidades. A partir desse planeamento, as tarefas foram divididas 
por áreas, de forma a permitir desenvolvimento em paralelo e reduzir 
dependências.

A distribuição principal de responsabilidades foi a seguinte:
\begin{itemize}
  \item Torres: lógica principal do jogo e integração dos sistemas.
  \item Danilo: construção do mapa e elementos associados ao cenário 
  (assets, organização do nível e interação com o ambiente).
  \item Guilherme: sistema de navegação e \textit{pathfinding}, bem como 
  suporte ao design e implementação do questionário.
\end{itemize}

Esta divisão permitiu manter uma estrutura de trabalho clara, onde cada 
membro tinha um foco definido, mantendo coordenação através do planeamento 
conjunto e da integração progressiva das funcionalidades.

\section{Tech Stack}

Esta secção descreve as principais tecnologias utilizadas no desenvolvimento do 
\textit{Casa dos Minerais}. A escolha da stack procurou equilibrar simplicidade, 
performance e facilidade de manutenção, garantindo uma base sólida para execução 
no navegador e expansão futura do projeto.

\subsection{Linguagem (Odin)}

O \textit{Casa dos Minerais} foi feito em Odin, uma linguagem compilada 
com foco em simplicidade, desempenho e controlo direto sobre recursos. 
A escolha de Odin permitiu implementar a lógica do jogo numa base leve, 
sem dependências excessivas, mantendo ao mesmo tempo um modelo de programação 
próximo de linguagens de baixo nível, adequado para aplicações interativas.

Outro fator relevante foi a boa integração de Odin com bibliotecas C, o que 
facilitou a utilização da raylib para gráficos e multimédia. Esta combinação 
simplifica a organização do projeto e torna mais fácil manter o código e 
adaptá-lo a diferentes plataformas no futuro, sem reescrever a lógica central.

\subsection{Biblioteca gráfica/multimédia (raylib)}

Para a execução nativa foi utilizada a raylib, integrada através do módulo 
\texttt{vendor:raylib}. Esta biblioteca fornece a base do ciclo de aplicação 
em desktop, incluindo criação de janela, controlo de \textit{frame rate}, 
câmara 2D, leitura de input do rato e desenho dos elementos visuais.

No código nativo (\texttt{native/main.odin}), a raylib é usada para inicializar 
e executar o \textit{loop} principal e para dar suporte à navegação no mapa. 
Exemplos de chamadas relevantes incluem \texttt{InitWindow} e 
\texttt{SetTargetFPS} (configuração), \texttt{BeginMode2D} e 
\texttt{GetScreenToWorld2D} (câmara e conversão de coordenadas), e funções de 
desenho como \texttt{DrawRectangle} e \texttt{DrawCircleV}. Esta abordagem 
permitiu validar rapidamente a lógica do jogo e o comportamento de interação 
num ambiente local simples de executar.

No estado atual do projeto, a raylib é usada sobretudo como base gráfica e de 
input no alvo nativo. Na versão web, a camada de áudio é gerida pelo JavaScript através da Web Audio API, 
o que permite adaptar a gestão de som às restrições e APIs do navegador.

\subsection{Build web / WebAssembly (WASM)}

Para disponibilizar o jogo no navegador, a lógica principal é compilada em Odin 
para o alvo \texttt{js\_wasm32}, gerando o módulo \texttt{index.wasm}. Este 
processo é automatizado através de \texttt{build.sh} e \texttt{build.bat}, que 
também copiam o \texttt{odin.js} (runtime de Odin) para a pasta \texttt{web/}.

De forma resumida, o fluxo de build web é:
\begin{enumerate}
  \item copiar \texttt{core/sys/wasm/js/odin.js} para \texttt{web/odin.js};
  \item compilar o projeto com \texttt{odin build . -target:js\_wasm32};
  \item gerar a saída em \texttt{web/index.wasm}.
\end{enumerate}

No arranque da aplicação web, o \texttt{odin.js} é carregado em 
\texttt{web/index.html} e o WASM é inicializado em \texttt{web/index.js} 
através de \texttt{odin.runWasm("index.wasm", ...)}. A partir desse ponto, 
o JavaScript gere a camada de interface e renderização em \textit{canvas}, 
enquanto a lógica do jogo permanece no módulo compilado em WASM.

Como o navegador impõe restrições de acesso a ficheiros locais para módulos 
WASM, a execução é feita através de um servidor HTTP estático na pasta 
\texttt{web/}, conforme descrito no \texttt{README.md}.

\subsection{Scripts/ferramentas de build e colaboração (Git, etc.)}

O desenvolvimento foi suportado por um conjunto de ferramentas e scripts para 
facilitar colaboração e tornar o processo de build reprodutível. Para controlo 
de versões foi utilizado Git, com alojamento no GitHub, permitindo trabalho em 
paralelo e histórico completo de alterações através de \textit{commits}. Esta 
abordagem simplificou a integração do trabalho do grupo e reduziu conflitos 
durante a implementação.

Para automatizar tarefas repetitivas, o repositório inclui scripts de build, 
nomeadamente \texttt{build.sh} e \texttt{build.bat}. Estes scripts tratam da 
compilação para WebAssembly e da preparação da pasta \texttt{web/}, incluindo 
a cópia do runtime \texttt{odin.js} e a geração do ficheiro \texttt{index.wasm}. 
Desta forma, o processo de compilação torna-se consistente entre sistemas 
operativos e reduz erros associados a passos manuais.

Adicionalmente, a execução local da versão web é feita através de um servidor 
HTTP estático (conforme descrito no \texttt{README.md}), contornando as 
restrições do navegador relativas ao carregamento de módulos WASM a partir do 
sistema de ficheiros. Esta organização permitiu testar rapidamente alterações 
e manter um fluxo de desenvolvimento estável ao longo do projeto.

\begin{figure}[H]
  \centering
  \includegraphics[width=\textwidth]{figuras/fluxo_build_execucao.png}
  \caption{Fluxo simplificado de build e execução da versão web do jogo, 
  desde o código-fonte e scripts de compilação até ao carregamento do WASM 
  no navegador.}
  \label{fig:fluxo-build-execucao}
\end{figure}



%%%

% \begin{figure}[!htb]
%   \centering
%   \caption{Exemplo de Figura flutuante (\textit{in}
%   \url{http://en.wikipedia.org/wiki/Class_diagram})}
%   \includegraphics[width=13cm]{exemploFig}
%   \label{fig:exemplofig}
% \end{figure}
%
% \begin{center}
%
%   \captionof{figure}{Exemplo de Figura não flutuante (\textit{in}
%   \url{http://en.wikipedia.org/wiki/Class_diagram})}
%   \includegraphics[width=\textwidth]{exemploFig}
%   %\includegraphics[width=0.5\textwidth]{exemploFig}
%   \label{fig:exemplofigFixa}
% \end{center}

% \lstinputlisting[language=java,abovecaptionskip=20pt,caption={Exemplo
%     de listagem de parte de um ficheiro Java na pasta
% listagens.},{label=lst:javaparcial},firstline=38,lastline=44]{listagens/quicksort.java}

% \noindent
% \begin{minipage}{\linewidth}
%   \captionof{table}{Uma tabela de exemplo não flutuante} % título da tabela
%   \begin{tabular}{l l c r} % duas colunas à esquerda (l l), uma ao
%     % centro (c) e uma à direita (r) (4 colunas)
%     \hline\hline %insere duas linhas horizontais
%     Caso & Método 1 & Método 2 & Método 3 \\ [0.5ex] % insere tabela
%     \hline % insere uma linha horizontal
%     1 & 50 & 837 & 970 \\ % insure corpo da tabela
%     2 & 47 & 877 & 230 \\
%     3 & 31 & 25 & 415 \\
%     4 & 35 & 144 & 2356 \\
%     5 & 45 & 300 & 556 \\ [1ex] % [1ex] adiciona espaço vertical
%     \hline %insere uma linha horizontal
%   \end{tabular}
%   \label{tab:casoFIXO} % é utilizada para referir a tabela no texto
%   \\ %% linha em braco
%   \textit{Nota}: uma nota. remover se não for necessária.\\
% \end{minipage}
%
% %baseado num exemplo em http://www1.maths.leeds.ac.uk/latex/TableHelp1.pdf
% \begin{table}[!htb]
%   \caption{Uma tabela de exemplo}
%   \begin{tabular}{l l c r} % duas colunas à esquerda (l l), uma ao
%     % centro (c) e uma à direita (r) (4 colunas)
%     \hline\hline %insere duas linhas horizontais
%     Caso & Método 1 & Método 2 & Método 3 \\ [0.5ex] % insere tabela
%     \hline % insere uma linha horizontal
%     1 & 50 & 837 & 970 \\ % insere corpo da tabela
%     2 & 47 & 877 & 230 \\
%     3 & 31 & 25 & 415 \\
%     4 & 35 & 144 & 2356 \\
%     5 & 45 & 300 & 556 \\ [1ex] % [1ex] adiciona espaço vertical
%     \hline %insere uma linha horizontal
%   \end{tabular}
%   \label{tab:caso} % é utilizada para referir a tabela no texto
%   \\ %% linha em braco
%   \textit{Nota}: uma nota. remover se não for necessária.\\
% \end{table}
