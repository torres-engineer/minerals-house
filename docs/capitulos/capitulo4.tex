\chapter{Conclusão e Trabalho Futuro}
\label{ch:conclusao}

O projeto ``Casa dos Minerais'' alcançou o seu objetivo principal: criar uma experiência educativa interativa que desmistifica a relação entre os minerais e os objetos do quotidiano. Através da combinação de uma tecnologia robusta (Odin/WASM) com a acessibilidade da web, foi possível desenvolver um protótipo funcional e visualmente apelativo, capaz de engajar o utilizador através da exploração e da descoberta.

\section{Síntese do trabalho realizado}

Ao longo deste projeto, a equipa implementou:
\begin{itemize}
    \item Um motor de jogo híbrido que corre nativamente e no navegador, garantindo portabilidade.
    \item Um sistema de mapas e camadas que permite criar cenários ricos e detalhados.
    \item Mecânicas de \textit{pathfinding} (A*) que permitem uma navegação fluida do personagem.
    \item Uma interface de utilizador funcional com menus, diálogos e feedback visual.
    \item Um sistema de questionários dinâmico que valida a aprendizagem do jogador.
    \item Conteúdo educativo estruturado em JSON, facilitando a expansão e tradução.
\end{itemize}

O resultado é um jogo que cumpre a promessa pedagógica de ligar a geologia à vida diária, demonstrando que é possível aprender conceitos complexos de forma lúdica.

\section{Limitações Atuais}

Devido ao âmbito temporal e académico do projeto, algumas funcionalidades idealizadas não puderam ser plenamente implementadas nesta versão:

\begin{enumerate}
    \item \textbf{Acessibilidade Completa:} Embora o jogo seja jogável, carece de suporte total para tecnologias de apoio (leitores de ecrã) e navegação exclusiva por teclado, limitando o seu alcance a utilizadores com deficiência.
    \item \textbf{Conteúdo Limitado:} O protótipo conta atualmente com dois níveis completos e funcionais. No entanto, a visão global do projeto incluía uma variedade maior de cenários, para abranger mais objetos e minerais.
    \item \textbf{Interação Tátil:} O suporte para dispositivos móveis (telas de toque) é funcional mas não otimizado, podendo apresentar dificuldades em ecrãs pequenos.
\end{enumerate}

\section{Trabalho Futuro}

O ``Casa dos Minerais'' foi desenhado como uma plataforma escalável. Embora a versão inicial se foque na experiência \textit{single-player} e na exploração da casa, a equipa já identificou várias expansões futuras que podem enriquecer significativamente a vertente pedagógica, a acessibilidade e a rejogabilidade.

Para além da potencial exploração de um modo \textbf{multijogador} (que poderia assumir a forma de questionários competitivos ou exploração cooperativa de um mapa), destacam-se as seguintes funcionalidades:

\begin{enumerate}
  \item {
      \textbf{Enciclopédia de Minerais (O ``Mineral-dex'')}
      \begin{longtable}{p{.3\textwidth}p{.7\textwidth}}
        Conceito & Implementar um ``caderno de campo'' ou
        enciclopédia digital.\\
        Funcionalidade & Cada vez que o jogador descobre um novo
        mineral num objeto, este seria automaticamente adicionado à
        enciclopédia. O jogador poderia consultá-la a qualquer
        momento para rever todos os minerais que já encontrou, as
        suas propriedades, proveniência, e uma lista dos objetos onde
        estes estão presentes (ex: ``Silício: encontrado nos
        microchips da TV e do telemóvel'').
      \end{longtable}
    }
  \item {
      \textbf{Níveis de Dificuldade Selecionáveis}
      \begin{longtable}{p{.3\textwidth}p{.7\textwidth}}
        Conceito & Alinhar o jogo ainda mais com os objetivos de
        acessibilidade, permitindo ao jogador adaptar a experiência
        às suas necessidades ou objetivos.\\
        Modos Propostos & {
          \begin{itemize}
            \item \textit{Modo Exploração (Fácil):} Focado puramente na
              descoberta. Os questionários seriam opcionais ou
              inexistentes, e não haveria temporizadores. Seria ideal
              para públicos mais jovens ou para uma utilização livre
              em sala de aula.
            \item \textit{Modo Desafio (Normal):} A experiência principal como
              definida, onde a conclusão do questionário de nível é
              obrigatória para progredir.
            \item \textit{Modo Contra-relógio (Difícil):} Ativaria o sistema
              de pontuação baseado no tempo, oferecendo um desafio
              adicional para quem procura rejogar os níveis.
          \end{itemize}
        }
      \end{longtable}
    }
  \item {
      \textbf{Sistema de Pistas (Hint System) nos Questionários}
      \begin{longtable}{p{.3\textwidth}p{.7\textwidth}}
        Conceito &  Integrar um sistema de ajuda nos minijogos de
        questionário para evitar a frustração e manter o foco na aprendizagem.\\
        Funcionalidade & O jogador poderia usar ``pontos'' (ganhos
        durante a exploração ou ao descobrir novos minerais) para
        pedir uma pista durante o questionário. As pistas poderiam
        incluir ``eliminar uma resposta errada''.
      \end{longtable}
    }
\end{enumerate}

Em suma, este projeto serviu como prova de conceito viável de que a tecnologia de jogos pode ser uma aliada poderosa na educação científica. Com o devido investimento futuro, o ``Casa dos Minerais'' tem potencial para se tornar uma ferramenta de referência no ensino da geologia.
