\chapter{Conclusão e Trabalho Futuro}
\label{ch:conclusao}

O ``Casa dos Minerais'' cumpriu o objetivo que se propôs: mostrar de
forma interativa a relação entre minerais e os objetos do dia a dia.
Usando Odin compilado para WASM e uma camada web em HTML/JS, o
resultado é um protótipo funcional que corre no navegador sem instalação.

\section{Síntese do trabalho realizado}

Ao longo deste projeto, a equipa implementou:
\begin{itemize}
  \item Um motor de jogo que corre nativamente e no navegador.
  \item Um sistema de mapas por camadas para compor os cenários.
  \item Navegação com \textit{pathfinding} (A*) para o personagem
    contornar obstáculos.
  \item Interface com menus, diálogos e feedback visual.
  \item Questionários gerados a partir dos dados JSON para validar a
    aprendizagem.
  \item Conteúdo educativo em JSON, separado do código, o que
    facilita a edição e tradução.
\end{itemize}

Na prática, o jogo liga a geologia ao dia a dia de uma forma que não
exige conhecimentos prévios.

\section{Limitações Atuais}
\label{sec:limitacoes}

Devido ao âmbito temporal e académico do projeto, algumas
funcionalidades idealizadas não puderam ser plenamente implementadas
nesta versão:

\begin{enumerate}
  \item \textbf{Acessibilidade Completa:} Embora o jogo seja jogável,
    carece de suporte total para tecnologias de apoio (leitores de
    ecrã) e navegação exclusiva por teclado, limitando o seu alcance
    a utilizadores com deficiência.
  \item \textbf{Conteúdo Limitado:} O protótipo conta atualmente com
    dois níveis completos e funcionais. No entanto, a visão global do
    projeto incluía uma variedade maior de cenários, para abranger
    mais objetos e minerais.
  \item \textbf{Interação Tátil:} O suporte para dispositivos móveis
    (telas de toque) é funcional mas não otimizado, podendo
    apresentar dificuldades em ecrãs pequenos.
\end{enumerate}

\section{Trabalho Futuro}
\label{sec:future}

O ``Casa dos Minerais'' foi desenhado como uma plataforma escalável.
Embora a versão inicial se foque na experiência
\textit{single-player} e na exploração da casa, a equipa já
identificou várias expansões futuras que podem enriquecer
significativamente a vertente pedagógica, a acessibilidade e a rejogabilidade.

Para além da potencial exploração de um modo \textbf{multijogador}
(que poderia assumir a forma de questionários competitivos ou
exploração cooperativa de um mapa), destacam-se as seguintes funcionalidades:

\begin{enumerate}
  \item {
      \textbf{Enciclopédia de Minerais (O ``Mineral-dex'')}
      
      \begin{tabular}{p{.3\textwidth}p{.6\textwidth}}
        \textbf{Conceito} & Implementar um ``caderno de campo'' ou
        enciclopédia digital.\\
        \textbf{Funcionalidade} & Cada vez que o jogador descobre um novo
        mineral num objeto, este seria automaticamente adicionado à
        enciclopédia. O jogador poderia consultá-la a qualquer
        momento para rever todos os minerais que já encontrou, as
        suas propriedades, proveniência, e uma lista dos objetos onde
        estes estão presentes (ex: ``Silício: encontrado nos
        microchips da TV e do telemóvel'').
      \end{tabular}
    }
  \item {
      \textbf{Níveis de Dificuldade Selecionáveis}
      
      \begin{tabular}{p{.3\textwidth}p{.6\textwidth}}
        \textbf{Conceito} & Alinhar o jogo ainda mais com os objetivos de
        acessibilidade, permitindo ao jogador adaptar a experiência
        às suas necessidades ou objetivos.\\
        \textbf{Modos Propostos} & {
          \begin{minipage}[t]{\linewidth}
          \begin{itemize}[leftmargin=*]
            \item \textit{Modo Exploração (Fácil):} Focado puramente na
              descoberta. Os questionários seriam opcionais ou
              inexistentes, e não haveria temporizadores. Seria ideal
              para públicos mais jovens ou para uma utilização livre
              em sala de aula.
            \item \textit{Modo Desafio (Normal):} A experiência principal como
              definida, onde a conclusão do questionário de nível é
              obrigatória para progredir.
            \item \textit{Modo Contra-relógio (Difícil):} Ativaria o sistema
              de pontuação baseado no tempo, oferecendo um desafio
              adicional para quem procura rejogar os níveis.
          \end{itemize}
          \end{minipage}
        }
      \end{tabular}
    }
  \item {
      \textbf{Sistema de Pistas (Hint System) nos Questionários}
      
      \begin{tabular}{p{.3\textwidth}p{.6\textwidth}}
        \textbf{Conceito} &  Integrar um sistema de ajuda nos minijogos de
        questionário para evitar a frustração e manter o foco na aprendizagem.\\
        \textbf{Funcionalidade} & O jogador poderia usar ``pontos'' (ganhos
        durante a exploração ou ao descobrir novos minerais) para
        pedir uma pista durante o questionário. As pistas poderiam
        incluir ``eliminar uma resposta errada''.
      \end{tabular}
    }
\end{enumerate}

O projeto mostra que é possível usar um jogo simples para ensinar
geologia de forma acessível. Com mais conteúdo e melhorias de
acessibilidade, o ``Casa dos Minerais'' pode vir a ser útil em mais
contextos educativos.
