% Para preencher 

\newcommand{\ESCOLA}{<<Nome da Escola>>}
%\newcommand{\ESCOLA}{Escola Superior de Tecnologia e Gestão}
%\newcommand{\ESCOLA}{Escola Superior de Educação}
%\newcommand{\ESCOLA}{Escola Superior Agrária}
%\newcommand{\ESCOLA}{Escola Superior de Saúde}


%\newcommand{\CURSO}{Licenciatura/Mestrado/... em ...}

\newcommand{\TITULO}{<<Título>>}

% Se pretender subtítulo
\newcommand{\SUBTITULO}{Opcionalmente, colocar aqui sub-título com máximo de vinte palavras}
% Se NÃO pretender subtítulo retire o % (comentário) da linha seguinte e coloque na linha anterior
%\newcommand{\SUBTITULO}{} 

\newcommand{\NOMEALUNO}{<<Nome Completo Estudante>>}

\newcommand{\DECLARACAO}{<<Dissertação apresentada ao instituto politécnico de beja para obtenção do grau de mestre em xx, ramo de xx>>}

\newcommand{\LOCAL}{<<LOCAL>>}
\newcommand{\MES}{<<MES>>}
\newcommand{\ANO}{<<ANO>>}


 %se for um estágio deve ser retirado o comentário da linha seguinte e indicar o orientador na entidade de acolhimento do estágio
%\newcommand{\ORIENTADORENTIDADE}{Título académico e nome do(a) orientador(a) na entidade de acolhimento, NomeDaEntidade, por exemplo "Eng. Nome Completo ou Abreviado}

\newcommand{\ORIENTADORIPBEJAA}{Colocar o título Académico e nome da(a) docente orientador(a), por exemplo "Doutora Nome Completo ou Abreviado"}
% se existir segundo orientador do IPBeja, retirar o comentário da linha seguinte
%\newcommand{\ORIENTADORIPBEJAB}{Colocar o título Académico e nome do(a) segundo(a) docente orientador(a), se existente} 


% retirar comentário e preencher se existente
%\newcommand{\DEDICATORIA}{ texto a colocar}
