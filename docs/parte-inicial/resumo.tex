\tituloPretextual{Resumo}

Este projeto apresenta o ``Casa dos Minerais'', um jogo educativo
desenvolvido para aumentar a literacia sobre a geologia no
quotidiano. A maioria das pessoas desconhece a proveniência dos
materiais que compõem os objetos comuns, como telemóveis ou
eletrodomésticos. O objetivo principal é tornar visível esta ligação
entre recursos minerais e produtos finais, utilizando uma abordagem
de aprendizagem por descoberta e gamificação.

Tecnicamente, o jogo utiliza uma arquitetura híbrida inovadora, com a
lógica central implementada na linguagem Odin e compilada para
WebAssembly (WASM), permitindo execução nativa no navegador web. A
interface gráfica combina renderização em Canvas com elementos HTML
semânticos para acessibilidade.

O protótipo permite ao utilizador explorar cenários familiares (como
uma casa), identificar objetos e consultar os minerais associados ao
seu fabrico. Para consolidar a aprendizagem, o sistema inclui
questionários de avaliação formativa no final de cada nível. Os
resultados demonstram que é possível criar ferramentas educativas
robustas e acessíveis que aproximam conceitos científicos abstratos
da realidade tangível dos alunos.

\vspace{1cm}

\textbf{Palavras-chave}: Gamificação, Educação, Geologia,
WebAssembly, Odin, Jogos Sérios, Minerais, Desenvolvimento Web.
