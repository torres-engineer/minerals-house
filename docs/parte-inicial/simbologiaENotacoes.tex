\tituloPretextual{Simbologia e Notações}

\begin{longtable}{p{.3\textwidth}p{.7\textwidth}} 

$\land$ (E lógico / Conjunção) & Representa a operação "E". Ex: $P \land Q$ (P e Q). \\
$\lor$ (OU lógico / Disjunção) & Representa a operação "OU". Ex: $P \lor Q$ (P ou Q). \\
$\lnot$ (Negação lógica) & Representa a operação "NÃO". Ex: $\lnot P$ (Não P). \\
$\rightarrow$ (Implicação) & "Se... então...". Ex: $P \rightarrow Q$ (Se P, então Q). \\
$\forall$ (Para todo) & Quantificador Universal. Indica que uma proposição é verdadeira para todos os elementos de um conjunto. \\
$\exists$ (Existe) & Quantificador Existencial. Indica que existe pelo menos um elemento para o qual a proposição é verdadeira. \\
$\in$ (Pertence a) & Indica que um elemento faz parte de um conjunto. Ex: $x \in S$. \\
$\notin$ (Não pertence a) & Indica que um elemento não faz parte de um conjunto. \\
$\subseteq$ (Subconjunto) & Indica que um conjunto está contido noutro. Ex: $A \subseteq B$. \\
$\cup$ (União de conjuntos) & Operação que combina todos os elementos de dois ou mais conjuntos. \\
$\cap$ (Interseção de conjuntos) & Operação que resulta nos elementos comuns a dois ou mais conjuntos.

\end{longtable}
