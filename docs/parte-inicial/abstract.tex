\tituloPretextual{Abstract}

This project presents ``Minerals' House'', an educational game
designed to improve literacy regarding geology in everyday life. Most
people are unaware of the origin of materials that make up common
objects, such as mobile phones or household appliances. The main goal
is to make the connection between mineral resources and final
products visible, using a discovery-based learning approach and gamification.

Technically, the game employs an innovative hybrid architecture, with
core logic implemented in the Odin language and compiled to
WebAssembly (WASM), allowing native execution within a web browser.
The graphical interface combines Canvas rendering with semantic HTML
elements for accessibility.

The prototype allows users to explore familiar settings (such as a
house), identify objects, and consult the minerals associated with
their manufacturing. To consolidate learning, the system includes
formative assessment quizzes at the end of each level. The results
demonstrate that it is possible to create robust and accessible
educational tools that bridge the gap between abstract scientific
concepts and the tangible reality of students.

\vspace{1cm}

\textbf{Keywords}: Gamification, Education, Geology, WebAssembly,
Odin, Serious Games, Minerals, Web Development.
