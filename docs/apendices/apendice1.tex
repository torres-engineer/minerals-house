\chapter{Exploração Laboral}
\label{ch:conficts}

Muitas das vezes, o processo laboral para a criação de um bem como
descrito tanto na \ref{ch:intro} como \ref{sec:minerals}, é feito em
condições tão percárias que não dá para esconder a palavra exploração.

Essa exploração muitas vezes é tão má que nem dá para imaginar. Sem
queremos dizer que existem explorações piores que outras porque todas
são ruins, temos os exemplos atuais dos conflitos armados na
República Democrática do Congo e no Sudão, onde ambos os trabalhadores destes
países, ricos nestas matérias-primas, estão a sofrer massacres,
sofrem, continuam pobres porque a riqueza saí de lá e vai ter às mãos
de monópolios imperialistas, que no final, apenas adicionam a sua
marca ao produto que ainda o vendem mais caro do que o que vale, sem mencionar o
sangue derramado para aquele produto existir. Quem participou na
produção do produto é capaz de nunca ter acesso ao produto final. A
desigualdade.

Algumas das extrações dos minerais também têm um alto impacto
ambiental (no sentido negativo), que também seria importante informar.

O objetivo do jogo não é ensinar sobre estes factos para criar um
sentimento de culpa no jogador, mas sim criar a consciência de que as
coisas realmente não acontecem por magia e que a única parte má do
processo é o jogador tirar dinheiro da sua carteira para adquirir o
bem de que necessita.

Até porque a responsabilidade não é individual. Apenas uma ação
coletiva internacionalista era capaz de mudar este modo de produção
malígno descrito em cima. Coletiva porque vários de nós
participamos nele de formas diferentes sem nos dar conta e
internacionalista porque o trabalho provavelmente foi feito um pedaço
ali e outro acolá. O \textit{made in} Portugal apenas diz em que foi
em Portugal que a produção do bem terminou.

É por isso que para este jogo ter a capacidade de criar um impacto
positivo, ele tem que ser acessivél para o máximo de gente possivél,
usando os métodos descritos em \ref{sec:other}.
