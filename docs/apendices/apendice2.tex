\chapter{Software livre e dados abertos}

Além objetivos principais descritos em \ref{sec:objectives}, é
importante também ter como objetivo o uso de software livre e o uso
de dados abertos. Duas filosofias que permitem uma real emancipação
de sabedoria, coisa importante para a escola e educação, embora
demasiadas vezes estas não estejam presentes. Fomentam a partilha do
conhecimento, a transparência do saber, e o acesso universal a educação.

\section{Software livre}

Permite \textbf{saber} como uma tecnologia funciona, \textbf{estudar}
o código, melhorar e partilhar as melhorias para todos.

\section{Dados abertos}

O saber tem de estar disponível para todos, para aprender e ensinar,
sem restrições de direitos de autor (no sentido de patentes), ou até
mesmo de software (com formatos de dados proprietários). Essas
restrições podem inibir ou até mesmo fazer com que “o aprender” não
seja possível.

Podemos no futuro contribuir para bases de conhecimento abertas como
o Wikidata da Wikimedia Foundation, e de depois, buscar essa
informação, e até informação ligada a essa (como fontes para validar
a informação) através de APIs como o Wikidata Query Service usando
\href{https://query.wikidata.org/#\%23Most\%20famous\%20child\%20of\%20a\%20librarian\%0A\%23Children\%20of\%20librarians\%20with\%20the\%20most\%20number\%20of\%20sitelinks\%20\%28as\%20a\%20proxy\%20for\%20fame\%29\%0ASELECT\%20\%3Fperson\%20\%3FpersonLabel\%20\%3FparentLabel\%20\%3Flinkcount\%20WHERE\%20\%7B\%0A\%20\%20\%20\%20\%3Fparent\%20wdt\%3AP106\%20wd\%3AQ182436\%20.\%0A\%20\%20\%20\%20\%3Fparent\%20wdt\%3AP40\%20\%3Fperson\%20.\%0A\%20\%20\%20\%20\%3Fperson\%20wikibase\%3Asitelinks\%20\%3Flinkcount\%20.\%0A\%20\%20SERVICE\%20wikibase\%3Alabel\%20\%7B\%20bd\%3AserviceParam\%20wikibase\%3Alanguage\%20\%22\%5BAUTO_LANGUAGE\%5D\%2Cmul\%2Cen\%2Cde\%2Ces\%2Car\%2Cfr\%22\%20\%7D\%0A\%7D\%0AGROUP\%20BY\%20\%3Flinkcount\%20\%3Fperson\%20\%3FpersonLabel\%20\%3Fparent\%20\%3FparentLabel\%0AORDER\%20BY\%20DESC\%28\%3Flinkcount\%29}{SPARQL}.
