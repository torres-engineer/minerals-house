\chapter{Software Livre e Dados Abertos na Educação}
\label{ap:open_source}

Para além dos objetivos pedagógicos específicos sobre geologia, o
projeto ``Casa dos Minerais'' alinha-se com os princípios do Software
Livre e dos Dados Abertos. Estas filosofias são fundamentais em
contexto educativo, pois promovem a transparência, a colaboração e o
acesso universal ao conhecimento.

\section{Software Livre}
A opção por ferramentas de código aberto (como a linguagem Odin, a
biblioteca Raylib e o ecossistema Web) não é apenas técnica, mas
ideológica. O software livre garante as liberdades essenciais de:
\begin{itemize}
  \item \textbf{Usar} o programa para qualquer fim;
  \item \textbf{Estudar} como o programa funciona e adaptá-lo às
    necessidades (o que permite aos alunos ver o código por trás do jogo);
  \item \textbf{Redistribuir} cópias para ajudar o próximo;
  \item \textbf{Melhorar} o programa e libertar as melhorias para o público.
\end{itemize}
Em educação, isto significa que a ferramenta de ensino não é uma
``caixa negra'', mas sim um recurso aberto que convida à curiosidade
e à apropriação tecnológica.

\section{Dados Abertos e Conhecimento Partilhado}
O conhecimento científico deve ser um bem comum, livre de restrições
artificiais. A utilização de formatos abertos (como JSON para os
dados do jogo) facilita a reutilização da informação por terceiros.
Futuramente, o projeto poderá integrar-se com bases de conhecimento
colaborativas, como o Wikidata. Através de consultas SPARQL, seria
possível enriquecer dinamicamente o conteúdo do jogo com dados
atualizados sobre minerais, proveniência e aplicações, conectando o
``Casa dos Minerais'' à vasta rede do, \textit{potencialmente},
Linked Open Data.
