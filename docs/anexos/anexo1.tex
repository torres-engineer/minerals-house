\chapter{Catálogo de Objetos e Minerais}
\label{an:catalog}

A tabela seguinte apresenta a relação completa entre os objetos implementados no nível ``Casa'', a respetiva categoria e os minerais associados, conforme definido na estrutura de dados do jogo. Esta informação serve de base para a geração dos questionários.

\begin{longtable}{p{0.25\textwidth} p{0.15\textwidth} p{0.55\textwidth}}
\caption{Lista de objetos interativos e minerais constituintes.} \label{tab:anexo_objetos} \\
\hline
\textbf{Objeto} & \textbf{Categoria} & \textbf{Minerais e Utilização} \\
\hline
\endfirsthead
\multicolumn{3}{c}%
{\tablename\ \thetable\ -- \textit{Continuação da página anterior}} \\
\hline
\textbf{Objeto} & \textbf{Categoria} & \textbf{Minerais e Utilização} \\
\hline
\endhead
\hline \multicolumn{3}{r}{\textit{Continua na próxima página}} \\
\endfoot
\hline
\endlastfoot

Fogão Elétrico & Cozinha & 
\textbullet\ Ferro (Corpo do fogão e resistências) \newline
\textbullet\ Crómio (Resistências) \newline
\textbullet\ Níquel (Revestimento) \\
\hline

Torradeira & Cozinha & 
\textbullet\ Ferro (Estrutura da torradeira) \newline
\textbullet\ Mica (Isolamento térmico) \newline
\textbullet\ Cobre (Fios elétricos) \\
\hline

Máquina de Lavar & Lavandaria & 
\textbullet\ Ferro (Tambor e motor) \newline
\textbullet\ Neodímio (Ímanes do motor) \newline
\textbullet\ Cobre (Fios do motor) \\
\hline

Televisão & Eletrónica & 
\textbullet\ Silício (Chips e processadores) \newline
\textbullet\ Terras Raras (Ecrã colorido) \newline
\textbullet\ Ouro (Conectores) \\
\hline

Computador & Eletrónica & 
\textbullet\ Silício (Processador e memória) \newline
\textbullet\ Lítio (Bateria) \newline
\textbullet\ Zinco (Proteção da bateria) \\
\hline

Piano & Música & 
\textbullet\ Ferro (Cordas do piano) \newline
\textbullet\ Cobre (Cordas graves) \newline
\textbullet\ Zinco (Pinos de afinação) \\
\end{longtable}
